% Compile with xelatex --shell-escape

\documentclass{beamer}

\usepackage[utf8]{inputenc}
\usetheme{Singapore}
\usepackage{xcolor}
\setbeamertemplate{footline}[frame number]

\usepackage{algorithmicx}
\usepackage{algpseudocode}

%\usepackage{mathlist}

\usepackage{amsmath,amssymb,amsthm}

\newtheorem{defi}{Définition}
\newtheorem{theo}{Theorem}
\newtheorem{coro}{Corollaire}
\newtheorem{lem}{Lemme}

%\usepackage{epic,eepic}
%\usepackage{pstricks, pst-tree}

%\usepackage{times}

\usepackage{ulem}
%\usepackage{crayola}

\usepackage{pstricks, pst-tree}
\usepackage{auto-pst-pdf}

%\usepackage{gnuplottex}
% This variables allows to conditional generate the Gnuplot figures.
%\def \gnuplotx {Generate the figures}

\usepackage{graphicx}
\usepackage{epstopdf}

\setbeamercovered{dynamic}

\def\bxi{\boldsymbol\xi}
\def\bepsilon{\boldsymbol\epsilon}
\def\bomega{\boldsymbol\xi}
\def\aff{\operatorname{aff}}
\def\DD{\mathcal{D}}
\def\KK{\mathcal{K}}
\def\RR{\mathcal{R}}
\def\KK{\mathcal{S}}

\newcommand{\tim}[1]{\;\; \mbox{#1} \;\;}

\title[SA vs SAA]{Stochastic optimization\\Stochastic dual dynamic programming}

\author[Fabian Bastin]{Fabian Bastin \\ \url{bastin@iro.umontreal.ca} \\ Université de Montréal -- CIRRELT -- IVADO -- Fin-ML}

\date{}

\begin{document}

\frame{\titlepage}

\begin{frame}
\frametitle{Context}

Based on Anthony Papavisiliou's slides \url{https://ap-rg.eu/wp-content/uploads/2020/05/SDDP.pdf}.

\mbox{}

Introduced by Perreira and Pinto, 1991.
First designed to solve hydro-scheduling problems.

\mbox{}

\begin{itemize}
\item
Stochastic dynamic problem with finite horizon.
\item
Continuous, finite dimensional state and control.
\item
Convex cost, additive over stages -- we focus here on linear costs.
\item
Discrete, independent noises
\end{itemize}

\mbox{}

Julia implementation: SDDP.jl

\end{frame}

\begin{frame}
\frametitle{The Nested L-Shaped Decomposition Subproblem}

For each stage $t = 1,\ldots, H-1$, scenario $k = 1,\ldots,\KK_t$, we consider the problem $NLSD(t,k)$
\begin{align}
\min_{x_{t,k}, \theta_{t,k}}\ & c_{t,k}^T x_{t,k} + \theta_{t,k} \notag \\
\mbox{s.t. } &
W_t x_{t,k} = h_{t,k} - T_{t-1,k} x_{t-1,a(k)}, (\pi_{t,k})  \notag \\
& D_{t,k,j}x_{t,k} \geq d_{t,k,j},\ j = 1,\ldots,r_{t,k}, (\rho_{t,k}) \\
& E_{t,k,j}x_{t,k} + \theta_{t,k} \geq e_{t,k,j},\ j = 1,\ldots, s_{t,k}, (\sigma_{t,k}) \\
&
\begin{aligned}
 x_{t,k} \geq 0.  \notag
\end{aligned}
\end{align}
\begin{itemize}
\item
$\KK_t$: number of distinct scenarios at stage $t$,
\item
$a(k)$: ancestor of scenario $k$ at stage $t - 1$,
\item
$x_{t-1,a(k)}$: current solution from $a(k)$,
\item
Constraints (1): feasibility cuts,
\item
Constraints (2): optimality cuts
\end{itemize}

\end{frame}

\begin{frame}
\frametitle{Nested L-Shaped Method}

Building block: $NLSD(t, k)$: problem at stage $t$, scenario $k$.
\begin{itemize}
	\item 
Repeated application of the $L$-shaped method.
	\item 
	Parent nodes send proposals for solutions to their children
nodes
\item
Child nodes send cuts to their parents
\item
There are different sequence procedures that tell in which
order the problems corresponding to different nodes in the
scenario tree are solved.
\end{itemize}

\end{frame}

\begin{frame}
\frametitle{Nested L-Shaped Method}

\begin{center}
	\begin{pspicture}(-0.5,0.5)(4.5,2.5)
		\psframe(0,0)(3,2)
		\psline[arrows=->](3,1)(5,1)
		\pcline[arrows=->](0,1)(-2,1)
		 \uput[l](-2,1){Cut}
		 \uput[d](-1,1){$t-1$, $a(k)$}
		 \uput[u](-1.3,1){$\pi_{t,k}$, $\rho_{t,k}$, $\sigma_{t,k}$}
		 \uput[r](5,1){Trial solution}
		 \uput[u](4.5,1){$x_{t,k}$}
		 \uput[d](4.2,1){$t+1$, $\DD_{t+1}(k)$}
       \rput(1.5,1){$NLSD(t, k)$}
	\end{pspicture}
\end{center}

\mbox{}

\begin{itemize}
	\item 
$a(k)$: ancestor of scenario $k$.
	\item 
$\DD_{t+1}(k)$: descendants of scenario $k$ in period $t + 1$.

\end{itemize}

\end{frame}

\begin{frame}
	\frametitle{Illustration}
	
	\begin{small}
	\begin{center}
		\psset{treemode=R, levelsep=3cm}
		\pstree[thislevelsep=0,edge=\ncline,linecolor=red]
		{\Tn} % root
		{\pstree[linecolor=black]{\TC~[tnpos=b, name=s0]{$x_1$}}{
				\pstree{\TC[name=s1a]}{
					\pstree[linestyle=dashed]{\TC[name=s2a]}
					{\Tdot[name=s4a] \Tdot[name=s4b]}            \pstree[linestyle=dashed]{\TC[name=s2b]}
					{\Tdot[name=s4c] \Tdot[name=s4d]}
				}
				\pstree{\TC[name=s1b]}{
					\pstree[linestyle=dashed]{\TC[name=s2c]}
					{\Tdot[name=s4e] \Tdot[name=s4f]}            
					\pstree[linestyle=dashed]{\TC[name=s2d]}
					{\Tdot[name=s4g] \Tdot[name=s4h]}
				}
			}
			% We add the stage identifiers.
			\pstree[edge = none, linecolor=black]{\Tr{First stage}}{
				\pstree{\Tr{Second stage}}{
					\pstree{\Tr{Third stage}}{\Tr{\ldots}}
				}
			}
		}
		\ncline[linestyle=dotted,nodesep=2pt]{s1a}{s1b}
		\ncline[linestyle=dotted,nodesep=2pt]{s2a}{s2b}
		\ncline[linestyle=dotted,nodesep=2pt]{s2c}{s2d}
		\ncline[linestyle=dotted,nodesep=2pt]{s4a}{s4b}
		\ncline[linestyle=dotted,nodesep=2pt]{s4c}{s4d}
		\ncline[linestyle=dotted,nodesep=2pt]{s4e}{s4f}
		\ncline[linestyle=dotted,nodesep=2pt]{s4g}{s4h}
		
	\end{center}
\end{small}

\begin{itemize}
	\item 
Node: $(t = 1, k = 1)$.
	\item 
Direction: forward.
	\item 
Output: $x_1$.
\end{itemize}	

\end{frame}

\begin{frame}
	\frametitle{Illustration}
	
	\begin{small}
		\begin{center}
			\psset{treemode=R, levelsep=3cm}
			\pstree[thislevelsep=0,edge=\ncline]
			{\Tn} % root
			{\pstree{\TC~[tnpos=b, name=s0]{$x_1$}}{
					\pstree[linecolor=black]{\TC[linecolor=red, name=s1a]}{
						\pstree[linestyle=dashed]{\TC[name=s2a]}
						{\Tdot[name=s4a] \Tdot[name=s4b]}            \pstree[linestyle=dashed]{\TC[name=s2b]}
						{\Tdot[name=s4c] \Tdot[name=s4d]}
					}
					\pstree[linecolor=black]{\TC[linecolor=red, name=s1b]}{
						\pstree[linestyle=dashed]{\TC[name=s2c]}
						{\Tdot[name=s4e] \Tdot[name=s4f]}            
						\pstree[linestyle=dashed]{\TC[name=s2d]}
						{\Tdot[name=s4g] \Tdot[name=s4h]}
					}
				}
				% We add the stage identifiers.
				\pstree[edge = none, linecolor=black]{\Tr{First stage}}{
					\pstree{\Tr{Second stage}}{
						\pstree{\Tr{Third stage}}{\Tr{\ldots}}
					}
				}
			}
			\ncline[linestyle=dotted,nodesep=2pt]{s1a}{s1b}
			\ncline[linestyle=dotted,nodesep=2pt]{s2a}{s2b}
			\ncline[linestyle=dotted,nodesep=2pt]{s2c}{s2d}
			\ncline[linestyle=dotted,nodesep=2pt]{s4a}{s4b}
			\ncline[linestyle=dotted,nodesep=2pt]{s4c}{s4d}
			\ncline[linestyle=dotted,nodesep=2pt]{s4e}{s4f}
			\ncline[linestyle=dotted,nodesep=2pt]{s4g}{s4h}
			
		\end{center}
	\end{small}
	
	\begin{itemize}
		\item 
		Node: $(t = 2, k)$, $k \in \{ 1,\ldots,k_2 \}$.
		\item 
		Direction: forward.
		\item 
		Output: $x_{2,k}$, $k \in \{ 1,\ldots,k_2 \}$.
	\end{itemize}	
	
\end{frame}

\begin{frame}
	\frametitle{Illustration}
	
	\begin{small}
		\begin{center}
			\psset{treemode=R, levelsep=3cm}
			\pstree[thislevelsep=0,edge=\ncline]
			{\Tn} % root
			{\pstree{\TC~[tnpos=b, name=s0]{$x_1$}}{
					\pstree[linecolor=black]{\TC[name=s1a]}{
						\pstree[linestyle=dashed]{\TC[linecolor=red, name=s2a]}
						{\Tdot[name=s4a] \Tdot[name=s4b]}            \pstree[linestyle=dashed]{\TC[linecolor=red, name=s2b]}
						{\Tdot[name=s4c] \Tdot[name=s4d]}
					}
					\pstree[linecolor=black]{\TC[name=s1b]}{
						\pstree[linestyle=dashed]{\TC[linecolor=red, name=s2c]}
						{\Tdot[name=s4e] \Tdot[name=s4f]}            
						\pstree[linestyle=dashed]{\TC[linecolor=red, name=s2d]}
						{\Tdot[name=s4g] \Tdot[name=s4h]}
					}
				}
				% We add the stage identifiers.
				\pstree[edge = none]{\Tr{First stage}}{
					\pstree{\Tr{Second stage}}{
						\pstree{\Tr{Third stage}}{\Tr{\ldots}}
					}
				}
			}
			\ncline[linestyle=dotted,nodesep=2pt]{s1a}{s1b}
			\ncline[linestyle=dotted,nodesep=2pt]{s2a}{s2b}
			\ncline[linestyle=dotted,nodesep=2pt]{s2c}{s2d}
			\ncline[linestyle=dotted,nodesep=2pt]{s4a}{s4b}
			\ncline[linestyle=dotted,nodesep=2pt]{s4c}{s4d}
			\ncline[linestyle=dotted,nodesep=2pt]{s4e}{s4f}
			\ncline[linestyle=dotted,nodesep=2pt]{s4g}{s4h}
			
		\end{center}
	\end{small}
	
	\begin{itemize}
		\item 
		Node: $(t = 3, k)$, $k \in \{ 1,\ldots,k_3 \}$.
		\item 
		Direction: forward.
		\item 
		Output: $x_{3,k}$, $k \in \{ 1,\ldots,k_3 \}$.
	\end{itemize}	
	
\end{frame}

\begin{frame}
	\frametitle{Illustration}
	
	\begin{small}
		\begin{center}
			\psset{treemode=R, levelsep=3cm}
			\pstree[thislevelsep=0,edge=\ncline]
			{\Tn} % root
			{\pstree{\TC~[tnpos=b, name=s0]{$x_1$}}{
					\pstree[linecolor=black]{\Tcircle[name=s1a]{$/$}}{
						\pstree[linestyle=dashed]{\TC[linecolor=red, name=s2a]}
						{\Tdot[name=s4a] \Tdot[name=s4b]}            \pstree[linestyle=dashed]{\TC[linecolor=red, name=s2b]}
						{\Tdot[name=s4c] \Tdot[name=s4d]}
					}
					\pstree[linecolor=black]{\Tcircle[name=s1b]{$/$}}{
						\pstree[linestyle=dashed]{\TC[linecolor=red, name=s2c]}
						{\Tdot[name=s4e] \Tdot[name=s4f]}            
						\pstree[linestyle=dashed]{\TC[linecolor=red, name=s2d]}
						{\Tdot[name=s4g] \Tdot[name=s4h]}
					}
				}
			}
			\ncline[linestyle=dotted,nodesep=2pt]{s1a}{s1b}
			\ncline[linestyle=dotted,nodesep=2pt]{s2a}{s2b}
			\ncline[linestyle=dotted,nodesep=2pt]{s2c}{s2d}
			\ncline[linestyle=dotted,nodesep=2pt]{s4a}{s4b}
			\ncline[linestyle=dotted,nodesep=2pt]{s4c}{s4d}
			\ncline[linestyle=dotted,nodesep=2pt]{s4e}{s4f}
			\ncline[linestyle=dotted,nodesep=2pt]{s4g}{s4h}
			
		\end{center}
	\end{small}
	
	\begin{itemize}
		\item 
		Node: $(t = 3, k)$, $k \in \{ 1,\ldots,k_3 \}$.
		\item 
		Direction: backward.
		\item 
		Output: $(\pi_{3,k}, \rho_{3,k}, \sigma_{3,k})$, $k \in \{ 1,\ldots,k_3 \}$.
	\end{itemize}	
	
\end{frame}

\begin{frame}
	\frametitle{Illustration}
	
	\begin{small}
		\begin{center}
			\psset{treemode=R, levelsep=3cm}
			\pstree[thislevelsep=0,edge=\ncline]
			{\Tn} % root
			{\pstree{\Tcircle[name=s0]{$/$}}{
					\pstree[linecolor=black]{\TC[linecolor=red, name=s1a]}{
						\pstree[linestyle=dashed]{\TC[name=s2a]}
						{\Tdot[name=s4a] \Tdot[name=s4b]}            \pstree[linestyle=dashed]{\TC[name=s2b]}
						{\Tdot[name=s4c] \Tdot[name=s4d]}
					}
					\pstree[linecolor=black]{\TC[linecolor=red, name=s1b]}{
						\pstree[linestyle=dashed]{\TC[name=s2c]}
						{\Tdot[name=s4e] \Tdot[name=s4f]}            
						\pstree[linestyle=dashed]{\TC[name=s2d]}
						{\Tdot[name=s4g] \Tdot[name=s4h]}
					}
				}
			}
			\ncline[linestyle=dotted,nodesep=2pt]{s1a}{s1b}
			\ncline[linestyle=dotted,nodesep=2pt]{s2a}{s2b}
			\ncline[linestyle=dotted,nodesep=2pt]{s2c}{s2d}
			\ncline[linestyle=dotted,nodesep=2pt]{s4a}{s4b}
			\ncline[linestyle=dotted,nodesep=2pt]{s4c}{s4d}
			\ncline[linestyle=dotted,nodesep=2pt]{s4e}{s4f}
			\ncline[linestyle=dotted,nodesep=2pt]{s4g}{s4h}
			
		\end{center}
	\end{small}
	
	\begin{itemize}
		\item 
		Node: $(t = 2, k)$, $k \in \{ 1,\ldots,k_2 \}$.
		\item 
		Direction: backward.
		\item 
		Output: $(\pi_{2,k}, \rho_{2,k}, \sigma_{2,k})$, $k \in \{ 1,\ldots,k_2 \}$.
	\end{itemize}	
	
\end{frame}

\begin{frame}
\frametitle{Feasibility cuts}

If $NLSD(t, k)$ is infeasible, we compute $\pi_{t,k}$ and $\rho_{t,k} \leq 0$ such that
\begin{align*}
\pi_{t,k}^T \left(h_{t,k} - T_{t-1,k} x_{t-1,a(k)}\right) + \rho_{t,k}^Td_{t,k} &> 0, \\
\pi_{t,k}^T W_t + \rho_{t,k}^TD_{t,k} &\leq 0.
\end{align*}
The following is a valid feasibility cut for $NLSD(t - 1, a(k))$:
$$
D_{t-1, a(k)} x_{t-1,a(k)} \geq d_{t-1, a(k)},
$$
where
\begin{align*}
D_{t-1, a(k)} = \pi_{t,k}^TT_{t-1,k},\\
d_{t-1, a(k)} = \pi_{t,k}^Th_{t,k} + \rho_{t,k}^Td_{t,k}. %\rho_{t,k}^Td_{t,k}
\end{align*}
	
\end{frame}

\begin{frame}
\frametitle{Optimality cuts}
	
Solve $NLSD(t, k)$ for $j = 1,\ldots, \KK_{t-1}$, and compute
\begin{align*}
E &= \sum_{k \in \DD_t(j)} p(k, t\,|\,j, t - 1) \pi_{t,k}^T T_{t-1,k}, \\
e &= \sum_{k \in \DD_t(j)} p(k, t\,|\,j, t - 1) \left( \pi_{t,k}^T h_{t,k}
+ \sum_{i=1}^{r_{t,k}} \rho_{t,k,i} d_{t,k,i} + \sum_{i=1}^{s_{t,k}} \sigma_{t,k,i} e_{t,k,i} \right).
\end{align*}

$\DD_t(j)$: descendants at stage $t$ of a scenario $j$ at period $t - 1$.

\mbox{}

Note:
$$
p(k, t\,|\,j, t - 1) = \frac{p_{k,t}}{p_{j,t-1}}.
$$

\end{frame}

\begin{frame}
\frametitle{Recombined scenario tree}
	
\begin{center}

\begin{minipage}{0.44\textwidth}
\begin{small}
\psset{treemode=R}
\pstree[thislevelsep=0,edge=\ncline]{\Tn}{
\pstree{\Tcircle[name=s0]{$1$}}{
	\pstree{\TCircle[name=s1a]{$1$}}{
		\TCircle[name=s2a]{$1$}
		\TCircle[name=s2b]{$2$}
	}
	\pstree{\Tcircle[name=s1b]{$2$}}{
		\TCircle[name=s2c]{$3$}
		\TCircle[name=s2d]{$4$}
	}
}}
\end{small}
\end{minipage}
$\longrightarrow \qquad$
\begin{minipage}{0.44\textwidth}
\begin{small}
\psset{treemode=R}
\pstree[thislevelsep=0,edge=\ncline]{\Tn}{
	\pstree{\Tcircle[name=s0]{$1$}}{
		\pstree{\TCircle[name=s1a]{$1$}}{
			\TCircle[name=s2a]{$1$}
		}
		\pstree{\Tcircle[name=s1b]{$2$}}{
			\TCircle[name=s2c]{$2$}
		}
}}
\ncline{s1b}{s2a}
\ncline{s1a}{s2c}
\end{small}
\end{minipage}
\begin{itemize}
	\item 
When can we recombine nodes?
	\item 
When can we assign the same value function $V_{t+1}(x)$ to
each node $k$ of stage $t$?
\end{itemize}

\end{center}

\end{frame}

\begin{frame}
\frametitle{Nested decomposition is non-scalable}

Assume
\begin{itemize}
\item
$H$ time steps, $M_t$ discrete outcomes in each stage.
\item
No infeasibility cuts.
\end{itemize}

\begin{center}
\begin{footnotesize}
\psset{treemode=R}
\pstree[thislevelsep=0,edge=\ncline]{\Tn}{
	\pstree{\Tcircle[name=s0]{$1$}}{
		\pstree{\TCircle[name=s1a]{$1$}}{
			\TCircle[name=s2a]{$1$}
			\TCircle[name=s2b]{$2$}
		}
		\pstree{\Tcircle[name=s1b]{$2$}}{
			\TCircle[name=s2c]{$3$}
			\TCircle[name=s2d]{$4$}
		}
}
\pstree[edge = none]{\Tr{$M_1$}}{
	\pstree{\Tr{$M_2$}}{\Tr{$M_3$}}
}
}
\ncline{s1a}{s2c}
\ncline{s1a}{s2d}
\ncline{s1b}{s2a}
\ncline{s1b}{s2b}
\end{footnotesize}
\end{center}

\begin{itemize}
	\item 
Forward pass: $M_1 + M_1 \times M_2 + \ldots = \sum_{t = 1}^H \prod_{j = 1}^t M_j$;
	\item 
Backward pass:  $\sum_{t = 2}^{H-1} \prod_{j = 1}^t M_j$.
\end{itemize}

\end{frame}

\begin{frame}
	\frametitle{Nested decomposition vs extended form}
	
	Alternative to nested decomposition is extended form.
	\begin{itemize}
		\item 
		Extended form will not even load in memory
		\item 
		Nested decomposition will load in memory, but will not
		terminate (for large problems)
	\end{itemize}
	
	Nested Decomposition lays the foundations for SDDP.
	
	\mbox{}
	
	As the method name suggests, we are in the context of dynamic programming.
	
\end{frame}

\begin{frame}
	\frametitle{Enumerating Versus Simulating}

\begin{center}
\psset{treemode=R}
\pstree[thislevelsep=0,edge=\ncline]{\Tn}{
	\pstree{\TC}{
		\pstree{\TCircle[name=s1a]{$1$}}{
			\TCircle[name=s2a]{$1$}
			\TCircle[name=s2b]{$2$}
		}
		\pstree{\Tcircle[name=s1b]{$2$}}{
			\TCircle[name=s2c]{$3$}
			\TCircle[name=s2d]{$4$}
		}
	}
}
\ncline{s1a}{s2c}
\ncline{s1a}{s2d}
\ncline{s1b}{s2a}
\ncline{s1b}{s2b}
\end{center}

\begin{itemize}
	\item 
	Enumeration: $\{(1, 1), (1, 2), (1, 3), (1, 4), (2, 1), (2, 2), (2,
		3), (2, 4))\}$
	\item 
	Simulation (with 3 samples): $\{(1, 3), (2, 1), (1, 4)\}$
\end{itemize}
\end{frame}

\begin{frame}
\frametitle{Making Nested Decomposition Scalable}

Solution for forward pass:
\begin{itemize}
	\item 
in the forward pass, we simulate instead of enumerating;
	\item 
this results in a probabilistic upper bound / termination
criterion;
\end{itemize}

\mbox{}

Solutions for backward pass
\begin{itemize}
	\item 
In the backward pass, we share cuts among nodes of the
same time period
\item
This requires an assumption of \textcolor{red}{stagewise independence}.
\end{itemize}

\mbox{}

\textcolor{red}{Stagewise independence}: the random vector $\xi_{t+1}$ is independent of $\xi_{[t]} = (\xi_1,\ldots, \xi_t)$ for $t = 1,\ldots, H-1$ (see for instance A. Shapiro, ``Analysis of stochastic dual dynamic programming'', European Journal of Operational Research, 209(1), 2011, pp. 63--72). Here, $\xi_1$ has a unique realization.

\end{frame}

\begin{frame}
	\frametitle{Implications for forward pass}

At each forward pass we solve $H-1$ NLSD problems

\begin{center}
	\psset{treemode=R}
	\pstree[thislevelsep=0,edge=\ncline]{\Tn}{
		\pstree{\TC[name=s0]}{
			\pstree{\TC[name=s1a]}{
				\TC[name=s2a]
				\TC[name=s2b]
			}
			\pstree{\TC[name=s1b]}{
				\TC[name=s2c]
				\TC[name=s2d]
			}
		}
	}
	\ncline{s1a}{s2c}
	\ncline{s1a}{s2d}
	\ncline{s1b}{s2a}
	\ncline{s1b}{s2b}
	\ncline[linewidth=3pt,linecolor=red]{s0}{s1b}
	\ncline[linewidth=3pt,linecolor=red]{s1b}{s2c}
\end{center}
 
For $K$ Monte Carlo simulations, we solve $1 + K\times (H-1)$ linear
programs.

\end{frame}

\begin{frame}
\frametitle{Implications for backward pass}

Serial independence implies same value function for all nodes
of stage $t$ $\Longrightarrow$ cut sharing
.

\begin{center}
	\psset{treemode=R}
	\pstree[thislevelsep=0,edge=\ncline]{\Tn}{
		\pstree{\TC[name=s0]}{
			\pstree{\Tcircle[name=s1a]{$/$}}{
				\TC[linecolor=red, name=s2a]
				\TC[linecolor=red, name=s2b]
			}
			\pstree{\Tcircle[name=s1b]{$/$}}{
				\TC[linecolor=red, name=s2c]
				\TC[linecolor=red, name=s2d]
			}
		}
	}
	\ncline{s1a}{s2c}
	\ncline{s1a}{s2d}
	\ncline{s1b}{s2a}
	\ncline{s1b}{s2b}
	\ncline{s0}{s1b}
	\ncline{s1b}{s2c}
\end{center}

For a given trial sequence $x_{t,k}$, we solve $\sum_{t=2}^{H} M_t$ linear programs.

For $K$ trial sequences, we solve $K\sum_{t=2}^{H} M_t$ linear programs.

\end{frame}

\begin{frame}
\frametitle{Stagewise independence is not necessary}

We can use dual multipliers in stage $t + 1$ for cuts in stage $t$
even without stagewise independence.

\mbox{}

However, each node in stage $t$ has a different value function, so
\begin{itemize}
	\item 
more memory consumption;
	\item 
more optimality cuts needed because we are
approximating more value functions.
\end{itemize}
With stagewise independence, we can
\begin{itemize}
	\item 
get rid of the scenario tree;
	\item 
work with continuous distribution of $\xi_t$.
\end{itemize}

\end{frame}

\begin{frame}
\frametitle{SDDP forward pass}

\begin{itemize}
	\item 
Solve $NLSD(1, 1)$. Let $x_{1,1}$ be the optimal solution.
Initialize $\hat{x}_{1,i} = x_{1,1}$ for $i = 1,\ldots,K$.
\item
For $t = 2,\ldots, H$, $i = 1,\ldots, K$,
\begin{itemize}
	\item 
sample a vector $h_{t,i}$ from the set $h_{t,k}$, $k = 1,\ldots,M_t$;
\item
solve the $NLSD(t, i)$ with trial decision $\hat{x}_{t-1,i}$;
\item
store the optimal solution as $\hat{x}_{t,i}$.
\end{itemize}
\end{itemize}

\end{frame}

\begin{frame}
	\frametitle{SDDP backward pass}

\begin{itemize}
	\item 
For $t = H, H - 1,\ldots, 2$
\begin{itemize}
	\item 
For $i = 1,\ldots, K$,
\begin{itemize}
	\item 
Repeat for $k = 1,\ldots, M_t$, solve $NLSD(t, k)$ with trial decision $\hat{x}_{t-1,i}$.
\item
Compute
\begin{align*}
E_{t-1} &= \sum_{k = 1}^{M_t} p_{t,k} \pi_{t,k,i}^T T_{t-1,k}, \\
e_{t-1} &= \sum_{k = 1}^{M_t} p_{t,k} \left( \pi_{t,k,i}^T h_{t,k} + \sigma_{t,k,i} e_{t,k} \right).
\end{align*}
Add the optimality cut
$$
E_{t-1}x_{t-1} + \theta_{t-1} \geq e_{t-1}
$$
to every $NLSD(t - 1, k)$, $k = 1,\ldots, M_{t-1}$.
\end{itemize}	
\end{itemize}	
\end{itemize}	

\end{frame}

\begin{frame}
\frametitle{Probabilistic upper bound}

Suppose we draw a sample $k$ of $(\xi_{t,k})$, $t = 1,\ldots,H$, and solve $NLSD(t, k)$ for $t = 1,\ldots,T$ (noting that for $t = 0$, the problem is deterministic).
\begin{itemize}
\item
This gives $x_{t,k}$, $t = 1,\ldots,H$ with a  cost $z_k = \sum_{t=1}^H c_k^Tx_k$.
\item
Consider $K$ independant samples, and the mean cost
$$
\overline{z}_K = \frac{1}{K} \sum_{k=1}^K z_k.
$$
By the Central Limit Theorem,
$$
\overline{z}_K \overset{\mathcal{D}}{\rightarrow} N(\mu, \sigma^2)
$$
\item
Each $(x_{t,k}, t = 1,\ldots,H)$ is feasible but not necessarily
optimal, so $\overline{z}_K$ is an estimate of an upper bound.
\end{itemize}

\end{frame}

\begin{frame}
\frametitle{Bounds and termination criterion}

After solving $NLSD(1, 1)$ in a forward pass we can compute a lower bound $z_{LB}$
as the objective function value of $NLDS(1, 1)$.

\mbox{}

Terminates if $z_{LB} \in (\overline{z}_K - \Phi^{-1}(1-\alpha/2) \sigma, \overline{z}_K + \Phi^{-1}(1-\alpha/2) \sigma)$, where $\Phi(\cdot)$ is the cumulative distribution of a Normal(0,1), and $\alpha$ is a confidence level set but the user. Typically, $\alpha = 0.05$.

\mbox{}

How to ensure $1\%$ optimality gap with $95\%$ confidence?
Choose $K$ such that $\Phi^{-1}(1-\alpha/2) \approx 0.01 \overline{z}_K$. Using the empirical standard deviation
$$
\hat{s} = \frac{1}{K-1} \sqrt{\sum_{k=1}^K (\overline{z}_K-z_k)^2},
$$
this leads to
$$
K = \left( \frac{\Phi^{-1}(1-\alpha/2)\hat{s}}{0.01 \overline{z}_K}\right)^2.
$$

\end{frame}

\begin{frame}
\frametitle{SDDP algorithm}

\begin{itemize}
\item
Set $K$, $\overline{z}_K = \infty$, $\hat{s} = 0$.
\item 
Forward pass, store $z_{LB}$ and $\overline{z}_K$.
Check termination.
\item
If the termination criterion is not met, do a backward pass.
\item 
Go to forward pass.
\end{itemize}

\end{frame}

\end{document}