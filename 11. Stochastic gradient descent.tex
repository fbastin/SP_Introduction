\documentclass[usepdftitle=false, aspectratio=169]{beamer}


\usepackage[utf8]{inputenc}
\usetheme{Singapore}
\usepackage{xcolor}
\setbeamertemplate{footline}[frame number]

\usepackage{algorithmicx}
\usepackage{algpseudocode}

%\usepackage{mathlist}

\usepackage{amsmath,amssymb,amsthm}

\newtheorem{defi}{Définition}
\newtheorem{theo}{Theorem}
\newtheorem{coro}{Corollaire}
\newtheorem{lem}{Lemme}

%\usepackage{epic,eepic}
%\usepackage{pstricks, pst-tree}

\usepackage{times}

\usepackage{ulem}
%\usepackage{crayola}

\usepackage{listings}
%\topmargin=-0.6in
\lstloadlanguages{C++}
\lstset{language=C++}
%%}

%\usepackage{gnuplottex}
% This variables allows to conditional generate the Gnuplot figures.
%\def \gnuplotx {Generate the figures}

\usepackage{graphicx}
\usepackage{epstopdf}

\setbeamercovered{dynamic}

% We add the following commonly used macros:

% vectors as boldsymbols:
\newcommand{\bsa}{{\boldsymbol{a}}}
\newcommand{\bsb}{{\boldsymbol{b}}}
\newcommand{\bsc}{{\boldsymbol{c}}}
\newcommand{\bsd}{{\boldsymbol{d}}}
\newcommand{\bse}{{\boldsymbol{e}}}
\newcommand{\bsf}{{\boldsymbol{f}}}
\newcommand{\bsg}{{\boldsymbol{g}}}
\newcommand{\bsh}{{\boldsymbol{h}}}
\newcommand{\bsi}{{\boldsymbol{i}}}
\newcommand{\bsj}{{\boldsymbol{j}}}
\newcommand{\bsk}{{\boldsymbol{k}}}
\newcommand{\bsl}{{\boldsymbol{l}}}
\newcommand{\bsm}{{\boldsymbol{m}}}
\newcommand{\bsn}{{\boldsymbol{n}}}
\newcommand{\bso}{{\boldsymbol{o}}}
\newcommand{\bsp}{{\boldsymbol{p}}}
\newcommand{\bsq}{{\boldsymbol{q}}}
\newcommand{\bsr}{{\boldsymbol{r}}}
\newcommand{\bss}{{\boldsymbol{s}}}
\newcommand{\bst}{{\boldsymbol{t}}}
\newcommand{\bsu}{{\boldsymbol{u}}}
\newcommand{\bsv}{{\boldsymbol{v}}}
\newcommand{\bsw}{{\boldsymbol{w}}}
\newcommand{\bsx}{{\boldsymbol{x}}}
\newcommand{\bsy}{{\boldsymbol{y}}}
\newcommand{\bsz}{{\boldsymbol{z}}}
\newcommand{\bsA}{{\boldsymbol{A}}}
\newcommand{\bsB}{{\boldsymbol{B}}}
\newcommand{\bsC}{{\boldsymbol{C}}}
\newcommand{\bsD}{{\boldsymbol{D}}}
\newcommand{\bsE}{{\boldsymbol{E}}}
\newcommand{\bsF}{{\boldsymbol{F}}}
\newcommand{\bsG}{{\boldsymbol{G}}}
\newcommand{\bsH}{{\boldsymbol{H}}}
\newcommand{\bsI}{{\boldsymbol{I}}}
\newcommand{\bsJ}{{\boldsymbol{J}}}
\newcommand{\bsK}{{\boldsymbol{K}}}
\newcommand{\bsL}{{\boldsymbol{L}}}
\newcommand{\bsM}{{\boldsymbol{M}}}
\newcommand{\bsN}{{\boldsymbol{N}}}
\newcommand{\bsO}{{\boldsymbol{O}}}
\newcommand{\bsP}{{\boldsymbol{P}}}
\newcommand{\bsQ}{{\boldsymbol{Q}}}
\newcommand{\bsR}{{\boldsymbol{R}}}
\newcommand{\bsS}{{\boldsymbol{S}}}
\newcommand{\bsT}{{\boldsymbol{T}}}
\newcommand{\bsU}{{\boldsymbol{U}}}
\newcommand{\bsV}{{\boldsymbol{V}}}
\newcommand{\bsW}{{\boldsymbol{W}}}
\newcommand{\bsX}{{\boldsymbol{X}}}
\newcommand{\bsY}{{\boldsymbol{Y}}}
\newcommand{\bsZ}{{\boldsymbol{Z}}}
% other commonly used boldsymbols:
\newcommand{\bsell}{{\boldsymbol{\ell}}}
\newcommand{\bszero}{{\boldsymbol{0}}} % vector of zeros
\newcommand{\bsone}{{\boldsymbol{1}}}  % vector of ones
% boldsymbol greeks:
\newcommand{\bsalpha}{{\boldsymbol{\alpha}}}
\newcommand{\bsbeta}{{\boldsymbol{\beta}}}
\newcommand{\bsgamma}{{\boldsymbol{\gamma}}}
\newcommand{\bsdelta}{{\boldsymbol{\delta}}}
\newcommand{\bsepsilon}{{\boldsymbol{\epsilon}}}
\newcommand{\bsvarepsilon}{{\boldsymbol{\varepsilon}}}
\newcommand{\bszeta}{{\boldsymbol{\zeta}}}
\newcommand{\bseta}{{\boldsymbol{\eta}}}
\newcommand{\bstheta}{{\boldsymbol{\theta}}}
\newcommand{\bsvartheta}{{\boldsymbol{\vartheta}}}
\newcommand{\bskappa}{{\boldsymbol{\kappa}}}
\newcommand{\bslambda}{{\boldsymbol{\lambda}}}
\newcommand{\bsmu}{{\boldsymbol{\mu}}}
\newcommand{\bsnu}{{\boldsymbol{\nu}}}
\newcommand{\bsxi}{{\boldsymbol{\xi}}}
\newcommand{\bspi}{{\boldsymbol{\pi}}}
\newcommand{\bsvarpi}{{\boldsymbol{\varpi}}}
\newcommand{\bsrho}{{\boldsymbol{\rho}}}
\newcommand{\bsvarrho}{{\boldsymbol{\varrho}}}
\newcommand{\bssigma}{{\boldsymbol{\sigma}}}
\newcommand{\bsvarsigma}{{\boldsymbol{\varsigma}}}
\newcommand{\bstau}{{\boldsymbol{\tau}}}
\newcommand{\bsupsilon}{{\boldsymbol{\upsilon}}}
\newcommand{\bsphi}{{\boldsymbol{\phi}}}
\newcommand{\bsvarphi}{{\boldsymbol{\varphi}}}
\newcommand{\bschi}{{\boldsymbol{\chi}}}
\newcommand{\bspsi}{{\boldsymbol{\psi}}}
\newcommand{\bsomega}{{\boldsymbol{\omega}}}
\newcommand{\bsGamma}{{\boldsymbol{\Gamma}}}
\newcommand{\bsDelta}{{\boldsymbol{\Delta}}}
\newcommand{\bsTheta}{{\boldsymbol{\Theta}}}
\newcommand{\bsLambda}{{\boldsymbol{\Lambda}}}
\newcommand{\bsXi}{{\boldsymbol{\Xi}}}
\newcommand{\bsPi}{{\boldsymbol{\Pi}}}
\newcommand{\bsSigma}{{\boldsymbol{\Sigma}}}
\newcommand{\bsUpsilon}{{\boldsymbol{\Upsilon}}}
\newcommand{\bsPhi}{{\boldsymbol{\Phi}}}
\newcommand{\bsPsi}{{\boldsymbol{\Psi}}}
\newcommand{\bsOmega}{{\boldsymbol{\Omega}}}

% Roman fonts:
\newcommand{\rma}{{\mathrm{a}}}
\newcommand{\rmb}{{\mathrm{b}}}
\newcommand{\rmc}{{\mathrm{c}}}
\newcommand{\rmd}{{\mathrm{d}}}
\newcommand{\rme}{{\mathrm{e}}}
\newcommand{\rmf}{{\mathrm{f}}}
\newcommand{\rmg}{{\mathrm{g}}}
\newcommand{\rmh}{{\mathrm{h}}}
\newcommand{\rmi}{{\mathrm{i}}}
\newcommand{\rmj}{{\mathrm{j}}}
\newcommand{\rmk}{{\mathrm{k}}}
\newcommand{\rml}{{\mathrm{l}}}
\newcommand{\rmm}{{\mathrm{m}}}
\newcommand{\rmn}{{\mathrm{n}}}
\newcommand{\rmo}{{\mathrm{o}}}
\newcommand{\rmp}{{\mathrm{p}}}
\newcommand{\rmq}{{\mathrm{q}}}
\newcommand{\rmr}{{\mathrm{r}}}
\newcommand{\rms}{{\mathrm{s}}}
\newcommand{\rmt}{{\mathrm{t}}}
\newcommand{\rmu}{{\mathrm{u}}}
\newcommand{\rmv}{{\mathrm{v}}}
\newcommand{\rmw}{{\mathrm{w}}}
\newcommand{\rmx}{{\mathrm{x}}}
\newcommand{\rmy}{{\mathrm{y}}}
\newcommand{\rmz}{{\mathrm{z}}}
\newcommand{\rmA}{{\mathrm{A}}}
\newcommand{\rmB}{{\mathrm{B}}}
\newcommand{\rmC}{{\mathrm{C}}}
\newcommand{\rmD}{{\mathrm{D}}}
\newcommand{\rmE}{{\mathrm{E}}}
\newcommand{\rmF}{{\mathrm{F}}}
\newcommand{\rmG}{{\mathrm{G}}}
\newcommand{\rmH}{{\mathrm{H}}}
\newcommand{\rmI}{{\mathrm{I}}}
\newcommand{\rmJ}{{\mathrm{J}}}
\newcommand{\rmK}{{\mathrm{K}}}
\newcommand{\rmL}{{\mathrm{L}}}
\newcommand{\rmM}{{\mathrm{M}}}
\newcommand{\rmN}{{\mathrm{N}}}
\newcommand{\rmO}{{\mathrm{O}}}
\newcommand{\rmP}{{\mathrm{P}}}
\newcommand{\rmQ}{{\mathrm{Q}}}
\newcommand{\rmR}{{\mathrm{R}}}
\newcommand{\rmS}{{\mathrm{S}}}
\newcommand{\rmT}{{\mathrm{T}}}
\newcommand{\rmU}{{\mathrm{U}}}
\newcommand{\rmV}{{\mathrm{V}}}
\newcommand{\rmW}{{\mathrm{W}}}
\newcommand{\rmX}{{\mathrm{X}}}
\newcommand{\rmY}{{\mathrm{Y}}}
\newcommand{\rmZ}{{\mathrm{Z}}}
% also commonly defined
\newcommand{\rd}{{\mathrm{d}}}
\newcommand{\ri}{{\mathrm{i}}}

% blackboards:
\newcommand{\bbA}{{\mathbb{A}}}
\newcommand{\bbB}{{\mathbb{B}}}
\newcommand{\bbC}{{\mathbb{C}}}
\newcommand{\bbD}{{\mathbb{D}}}
\newcommand{\bbE}{{\mathbb{E}}}
\newcommand{\bbF}{{\mathbb{F}}}
\newcommand{\bbG}{{\mathbb{G}}}
\newcommand{\bbH}{{\mathbb{H}}}
\newcommand{\bbI}{{\mathbb{I}}}
\newcommand{\bbJ}{{\mathbb{J}}}
\newcommand{\bbK}{{\mathbb{K}}}
\newcommand{\bbL}{{\mathbb{L}}}
\newcommand{\bbM}{{\mathbb{M}}}
\newcommand{\bbN}{{\mathbb{N}}}
\newcommand{\bbO}{{\mathbb{O}}}
\newcommand{\bbP}{{\mathbb{P}}}
\newcommand{\bbQ}{{\mathbb{Q}}}
\newcommand{\bbR}{{\mathbb{R}}}
\newcommand{\bbS}{{\mathbb{S}}}
\newcommand{\bbT}{{\mathbb{T}}}
\newcommand{\bbU}{{\mathbb{U}}}
\newcommand{\bbV}{{\mathbb{V}}}
\newcommand{\bbW}{{\mathbb{W}}}
\newcommand{\bbX}{{\mathbb{X}}}
\newcommand{\bbY}{{\mathbb{Y}}}
\newcommand{\bbZ}{{\mathbb{Z}}}
% commonly used shortcuts:
\newcommand{\C}{{\mathbb{C}}} % complex numbers
\newcommand{\F}{{\mathbb{F}}} % field, finite field
\newcommand{\N}{{\mathbb{N}}} % natural numbers {1, 2, ...}
\newcommand{\Q}{{\mathbb{Q}}} % rationals
\newcommand{\R}{{\mathbb{R}}} % reals
\newcommand{\Z}{{\mathbb{Z}}} % integers
% more commonly used shortcuts:
\newcommand{\CC}{{\mathbb{C}}} % complex numbers
\newcommand{\FF}{{\mathbb{F}}} % field, finite field
\newcommand{\NN}{{\mathbb{N}}} % natural numbers {1, 2, ...}
\newcommand{\QQ}{{\mathbb{Q}}} % rationals
\newcommand{\RR}{{\mathbb{R}}} % reals
\newcommand{\ZZ}{{\mathbb{Z}}} % integers
% more commonly used shortcuts:
\newcommand{\EE}{{\mathbb{E}}}
\newcommand{\PP}{{\mathbb{P}}}
\newcommand{\TT}{{\mathbb{T}}}
\newcommand{\VV}{{\mathbb{V}}}
% and even more commonly used shortcuts:
\newcommand{\Complex}{{\mathbb{C}}}
\newcommand{\Integer}{{\mathbb{Z}}}
\newcommand{\Natural}{{\mathbb{N}}}
\newcommand{\Rational}{{\mathbb{Q}}}
\newcommand{\Real}{{\mathbb{R}}}

%%% Issue with miktex
% indicator boldface 1:
%\DeclareSymbolFont{bbold}{U}{bbold}{m}{n}
%\DeclareSymbolFontAlphabet{\mathbbold}{bbold}
%\newcommand{\ind}{{\mathbbold{1}}}
%\newcommand{\bbone}{{\mathbbold{1}}}


% calligraphic letters:
\newcommand{\cala}{{\mathcal{a}}}
\newcommand{\calb}{{\mathcal{b}}}
\newcommand{\calc}{{\mathcal{c}}}
\newcommand{\cald}{{\mathcal{d}}}
\newcommand{\cale}{{\mathcal{e}}}
\newcommand{\calf}{{\mathcal{f}}}
\newcommand{\calg}{{\mathcal{g}}}
\newcommand{\calh}{{\mathcal{h}}}
\newcommand{\cali}{{\mathcal{i}}}
\newcommand{\calj}{{\mathcal{j}}}
\newcommand{\calk}{{\mathcal{k}}}
\newcommand{\call}{{\mathcal{l}}}
\newcommand{\calm}{{\mathcal{m}}}
\newcommand{\caln}{{\mathcal{n}}}
\newcommand{\calo}{{\mathcal{o}}}
\newcommand{\calp}{{\mathcal{p}}}
\newcommand{\calq}{{\mathcal{q}}}
\newcommand{\calr}{{\mathcal{r}}}
\newcommand{\cals}{{\mathcal{s}}}
\newcommand{\calt}{{\mathcal{t}}}
\newcommand{\calu}{{\mathcal{u}}}
\newcommand{\calv}{{\mathcal{v}}}
\newcommand{\calw}{{\mathcal{w}}}
\newcommand{\calx}{{\mathcal{x}}}
\newcommand{\caly}{{\mathcal{y}}}
\newcommand{\calz}{{\mathcal{z}}}
\newcommand{\calA}{{\mathcal{A}}}
\newcommand{\calB}{{\mathcal{B}}}
\newcommand{\calC}{{\mathcal{C}}}
\newcommand{\calD}{{\mathcal{D}}}
\newcommand{\calE}{{\mathcal{E}}}
\newcommand{\calF}{{\mathcal{F}}}
\newcommand{\calG}{{\mathcal{G}}}
\newcommand{\calH}{{\mathcal{H}}}
\newcommand{\calI}{{\mathcal{I}}}
\newcommand{\calJ}{{\mathcal{J}}}
\newcommand{\calK}{{\mathcal{K}}}
\newcommand{\calL}{{\mathcal{L}}}
\newcommand{\calM}{{\mathcal{M}}}
\newcommand{\calN}{{\mathcal{N}}}
\newcommand{\calO}{{\mathcal{O}}}
\newcommand{\calP}{{\mathcal{P}}}
\newcommand{\calQ}{{\mathcal{Q}}}
\newcommand{\calR}{{\mathcal{R}}}
\newcommand{\calS}{{\mathcal{S}}}
\newcommand{\calT}{{\mathcal{T}}}
\newcommand{\calU}{{\mathcal{U}}}
\newcommand{\calV}{{\mathcal{V}}}
\newcommand{\calW}{{\mathcal{W}}}
\newcommand{\calX}{{\mathcal{X}}}
\newcommand{\calY}{{\mathcal{Y}}}
\newcommand{\calZ}{{\mathcal{Z}}}

% Euler fraks:
\newcommand{\fraka}{{\mathfrak{a}}}
\newcommand{\frakb}{{\mathfrak{b}}}
\newcommand{\frakc}{{\mathfrak{c}}}
\newcommand{\frakd}{{\mathfrak{d}}}
\newcommand{\frake}{{\mathfrak{e}}}
\newcommand{\frakf}{{\mathfrak{f}}}
\newcommand{\frakg}{{\mathfrak{g}}}
\newcommand{\frakh}{{\mathfrak{h}}}
\newcommand{\fraki}{{\mathfrak{i}}}
\newcommand{\frakj}{{\mathfrak{j}}}
\newcommand{\frakk}{{\mathfrak{k}}}
\newcommand{\frakl}{{\mathfrak{l}}}
\newcommand{\frakm}{{\mathfrak{m}}}
\newcommand{\frakn}{{\mathfrak{n}}}
\newcommand{\frako}{{\mathfrak{o}}}
\newcommand{\frakp}{{\mathfrak{p}}}
\newcommand{\frakq}{{\mathfrak{q}}}
\newcommand{\frakr}{{\mathfrak{r}}}
\newcommand{\fraks}{{\mathfrak{s}}}
\newcommand{\frakt}{{\mathfrak{t}}}
\newcommand{\fraku}{{\mathfrak{u}}}
\newcommand{\frakv}{{\mathfrak{v}}}
\newcommand{\frakw}{{\mathfrak{w}}}
\newcommand{\frakx}{{\mathfrak{x}}}
\newcommand{\fraky}{{\mathfrak{y}}}
\newcommand{\frakz}{{\mathfrak{z}}}
\newcommand{\frakA}{{\mathfrak{A}}}
\newcommand{\frakB}{{\mathfrak{B}}}
\newcommand{\frakC}{{\mathfrak{C}}}
\newcommand{\frakD}{{\mathfrak{D}}}
\newcommand{\frakE}{{\mathfrak{E}}}
\newcommand{\frakF}{{\mathfrak{F}}}
\newcommand{\frakG}{{\mathfrak{G}}}
\newcommand{\frakH}{{\mathfrak{H}}}
\newcommand{\frakI}{{\mathfrak{I}}}
\newcommand{\frakJ}{{\mathfrak{J}}}
\newcommand{\frakK}{{\mathfrak{K}}}
\newcommand{\frakL}{{\mathfrak{L}}}
\newcommand{\frakM}{{\mathfrak{M}}}
\newcommand{\frakN}{{\mathfrak{N}}}
\newcommand{\frakO}{{\mathfrak{O}}}
\newcommand{\frakP}{{\mathfrak{P}}}
\newcommand{\frakQ}{{\mathfrak{Q}}}
\newcommand{\frakR}{{\mathfrak{R}}}
\newcommand{\frakS}{{\mathfrak{S}}}
\newcommand{\frakT}{{\mathfrak{T}}}
\newcommand{\frakU}{{\mathfrak{U}}}
\newcommand{\frakV}{{\mathfrak{V}}}
\newcommand{\frakW}{{\mathfrak{W}}}
\newcommand{\frakX}{{\mathfrak{X}}}
\newcommand{\frakY}{{\mathfrak{Y}}}
\newcommand{\frakZ}{{\mathfrak{Z}}}
% sets as Euler fraks:
\newcommand{\seta}{{\mathfrak{a}}}
\newcommand{\setb}{{\mathfrak{b}}}
\newcommand{\setc}{{\mathfrak{c}}}
\newcommand{\setd}{{\mathfrak{d}}}
\newcommand{\sete}{{\mathfrak{e}}}
\newcommand{\setf}{{\mathfrak{f}}}
\newcommand{\setg}{{\mathfrak{g}}}
\newcommand{\seth}{{\mathfrak{h}}}
\newcommand{\seti}{{\mathfrak{i}}}
\newcommand{\setj}{{\mathfrak{j}}}
\newcommand{\setk}{{\mathfrak{k}}}
\newcommand{\setl}{{\mathfrak{l}}}
\newcommand{\setm}{{\mathfrak{m}}}
\newcommand{\setn}{{\mathfrak{n}}}
\newcommand{\seto}{{\mathfrak{o}}}
\newcommand{\setp}{{\mathfrak{p}}}
\newcommand{\setq}{{\mathfrak{q}}}
\newcommand{\setr}{{\mathfrak{r}}}
\newcommand{\sets}{{\mathfrak{s}}}
\newcommand{\sett}{{\mathfrak{t}}}
\newcommand{\setu}{{\mathfrak{u}}}
\newcommand{\setv}{{\mathfrak{v}}}
\newcommand{\setw}{{\mathfrak{w}}}
\newcommand{\setx}{{\mathfrak{x}}}
\newcommand{\sety}{{\mathfrak{y}}}
\newcommand{\setz}{{\mathfrak{z}}}
\newcommand{\setA}{{\mathfrak{A}}}
\newcommand{\setB}{{\mathfrak{B}}}
\newcommand{\setC}{{\mathfrak{C}}}
\newcommand{\setD}{{\mathfrak{D}}}
\newcommand{\setE}{{\mathfrak{E}}}
\newcommand{\setF}{{\mathfrak{F}}}
\newcommand{\setG}{{\mathfrak{G}}}
\newcommand{\setH}{{\mathfrak{H}}}
\newcommand{\setI}{{\mathfrak{I}}}
\newcommand{\setJ}{{\mathfrak{J}}}
\newcommand{\setK}{{\mathfrak{K}}}
\newcommand{\setL}{{\mathfrak{L}}}
\newcommand{\setM}{{\mathfrak{M}}}
\newcommand{\setN}{{\mathfrak{N}}}
\newcommand{\setO}{{\mathfrak{O}}}
\newcommand{\setP}{{\mathfrak{P}}}
\newcommand{\setQ}{{\mathfrak{Q}}}
\newcommand{\setR}{{\mathfrak{R}}}
\newcommand{\setS}{{\mathfrak{S}}}
\newcommand{\setT}{{\mathfrak{T}}}
\newcommand{\setU}{{\mathfrak{U}}}
\newcommand{\setV}{{\mathfrak{V}}}
\newcommand{\setW}{{\mathfrak{W}}}
\newcommand{\setX}{{\mathfrak{X}}}
\newcommand{\setY}{{\mathfrak{Y}}}
\newcommand{\setZ}{{\mathfrak{Z}}}

% other commonly defined commands:
\newcommand{\wal}{{\rm wal}}
\newcommand{\floor}[1]{\left\lfloor #1 \right\rfloor} % floor
\newcommand{\ceil}[1]{\left\lceil #1 \right\rceil}    % ceil
\DeclareMathOperator{\cov}{Cov}
\DeclareMathOperator{\var}{Var}
\providecommand{\argmin}{\operatorname*{argmin}}
\providecommand{\argmax}{\operatorname*{argmax}}

\def\bepsilon{\boldsymbol\epsilon}
\def\bomega{\boldsymbol\xi}
\def\aff{\operatorname{aff}}

\newcommand{\tim}[1]{\;\; \mbox{#1} \;\;}

\title[SA vs SAA]{Stochastic optimization\\Stochastic gradient descent}

\author[Fabian Bastin]{Fabian Bastin (\url{bastin@iro.umontreal.ca}) \\ Université de Montréal -- CIRRELT -- IVADO -- Fin-ML}

\date{}

\begin{document}

%% AJOUTER DISCUSSION BIRGE ET LOUVEAUX P128

\frame{\titlepage}

\begin{frame}
\frametitle{Motivation}

The recent years have seen a huge success of machine learning, and a renew interest in the stochastic gradient algorithm, and the development of various variants.

\mbox{}

The stochastic gradient algorithm is a special case of the stochastic approximation method, which was first introduced in 1951, and can be seen as an alternative to the sample-path approach.

\mbox{}

The first part of these slides is based on
S. Bhatnagar, H.L. Prasad, L.A. Prashanth, ``Stochastic Recursive Algorithms for Optimization
Simultaneous Perturbation Methods'', Springer-Verlag, 2013.

\end{frame}

\begin{frame}
\frametitle{Robbins–Monro algorithm}

Introduced in 1951, initially as a root-finding problem.

\mbox{}

It can be easily extended to unconstrained optimization using first-order condition as a necessary condition to the problem
$$
\min_{x \in \RR^d} \ f(x)
$$
is
$$
\nabla_x f(x) = 0.
$$

\mbox{}

We therefore search for a zero of the gradient of $f(x)$.

\mbox{}

We can also restrain the feasible domain to $\Theta \subseteq \RR^d$:
$$
\min_{x \in \Theta} f(x) = \EE[Y(x,\xi)]
$$
In this case, we have to assume that $f$ reaches its minimum in the interior of $\Theta$.

\end{frame}

\begin{frame}
\frametitle{Problem in expectation}

We consider here
$$
f(x) = \EE[Y(x,\xi)]
$$
where the support of $\xi$ is $\Xi$.

\mbox{}

The problem to solve is
$$
\nabla_x \EE[Y(x,\xi)] = 0
$$
We assume here that we can exchange the expectation and derivation operators, i.e.
$$
\nabla_x \EE[Y(x,\xi)] = \EE[\nabla_x Y(x,\xi)]
$$

\end{frame}

\begin{frame}
\frametitle{Stochastic approximation}

Also know as as the \textcolor{red}{stochastic gradient descent (SGD) method}.

\mbox{}

\begin{algorithmic}
\State Choose a starting point $x_1$.
\State $k \leftarrow 1$
\While{Stopping criteria not satisfied}
\State Draw $\xi_k$ from $\bsxi$.
\State Select a step length $\alpha_k$.
\State Compute
$$
x_{k+1} = x_k - \alpha_k \nabla_x Y(x_k, \xi_k).
$$
\State $k \leftarrow k+1$
\EndWhile
\end{algorithmic}

\mbox{}

$\alpha_k$ is also called the \textcolor{red}{learning rate}.

\end{frame}

\begin{frame}
\frametitle{Assumptions}

% Robbins and Monro proved[3], Theorem 2 that {\displaystyle \theta _{n}}\theta_n converges in {\displaystyle L^{2}}L^{2} (and hence also in probability) to {\displaystyle \theta }\theta , and Blum[4] later proved the convergence is actually with probability one, provided that:

Assume a unique minimizer $x^*$, and
\begin{enumerate}
\item[A.1]
$f(x)$ is continuously differentiable and its gradient is Lipschitz continuous with Lipschitz constant $L > 0$, i.e. $\forall\, x, y \in \RR^d$,
$$
\| \nabla_x f(x) - \nabla_x f(y) \|_2 \leq L \| x - y \|_2
$$
\item[A.2]
The iterates remain a.s. bounded, i.e.
$$
\sup_k \| x_k \| < \infty\ \text{almost surely}.
$$
\item[A.3]
There exist scalars $M \geq 0$ and $M_V \geq 0$ s.t. $\forall\, k$,
$$
\var[\nabla_x Y(x, \xi_k)] \leq M + M_V\| \nabla_x f(x_k)\|_2^2
$$
\item[A.4]
The sequence $\alpha_k$, $k=1,2,\ldots$, satisfies
$$
\sum_{k=0}^{\infty}\alpha_k=\infty \quad \mbox{ and } \quad \sum_{k=0}^{\infty}\alpha_k^{2} < \infty.
$$
\end{enumerate}

\end{frame}

\begin{frame}
\frametitle{Assumptions: notes}

\begin{itemize}
\item 
In A.3, $\var_{\xi_k} [\nabla_x Y(x, \xi_k)]$ does not refer to the covariance matrix of $\nabla_x Y(x,  \xi_k)$.
\item
Variance of a random vector $g(\xi_k)$:
\begin{align*}
\var_{\xi_k} [g] &= \EE_{\xi_k} \left[ \left \| g - \EE_{\xi_k} [g] \right \|^2 \right] \\
& = \EE_{\xi_k} \left[ \| g \|^2 \right] - \left( \EE_{\xi_k} [ \| [g] \| ] \right)^2.
\end{align*}
\item
A well-known consequence of the Lipschitz continuity assumption A.1 is
$$
f(x) \leq f(y) + \nabla f(y)^T ( x - y ) + \frac{L}{2} \| x - y \|^2_2,
$$
$\forall\, x, y \in \RR^d$.
\end{itemize}

\end{frame}

\begin{frame}
\frametitle{Step lengths}

Consider the sequence of step lengths, also called \textcolor{blue}{positive gains sequence} $\{ \alpha_k \,|\, k \geq 1 \}$.

\mbox{}

This sequence satisfies the previous assumption in particular with
\begin{itemize}
\item 
$\alpha_k = \alpha/k$, given $\alpha > 0$.
\item 
$\alpha_k = \alpha/k^{\beta}$, $\forall\, k \geq 1$, given $\alpha > 0$ and $\beta \in (0.5, 1)$.
\item
$\alpha_k = \alpha(\ln k)/k$, $\forall\, k \geq 2$, given $\alpha_1 = \alpha > 0$.
\item
$\alpha_k = \alpha/(k \ln k)$, $\forall\, k \geq 2$, given $\alpha_1 = \alpha > 0$.
\end{itemize}

\end{frame}

\begin{frame}
\frametitle{Properties}

\begin{itemize}
\item 
Very cheap iteration: gradient w.r.t. just one observation. No function evaluation.
\item
Reminder: $d$ is a descent direction for $f$ at $x$ if
$$
d^T \nabla_x f(x) < 0
$$
\item
SGD is not a descent method as we can have $-\nabla_x Y(x,\xi_i)^T \EE[\nabla_x Y(x,\xi)] \geq 0$ with $\nabla_x f(x) \ne 0$.
\item
Descent in expectation: if $\nabla_x f(x) \ne 0$,
\begin{align*}
& \EE[-\nabla_x Y(x,\xi_i)^T \EE[\nabla_x Y(x,\xi)]] \\
&= - \nabla_x \EE[Y(x,\xi_i)]^T \nabla_x \EE[Y(x,\xi)] \\
&= - \nabla_x f(x)^T\nabla_x f(x) < 0
\end{align*}
\end{itemize}

%{\textstyle N(\theta )}{\textstyle N(\theta )} is uniformly bounded,
%{\textstyle M(\theta )}{\textstyle M(\theta )} is nondecreasing,
%{\textstyle M'(\theta ^{*})}{\textstyle M'(\theta ^{*})} exists and is positive, and

%Complexity results
%If {\textstyle f(\theta )}{\textstyle f(\theta )} is twice continuously differentiable, and strongly convex, and the minimizer of {\textstyle f(\theta )}{\textstyle f(\theta )} belongs to the interior of {\textstyle \Theta }{\textstyle \Theta }, then the Robbins–Monro algorithm will achieve the asymptotically optimal convergence rate, with respect to the objective function, being {\textstyle \operatorname {E} [f(\theta _{n})-f^{*}]=O(1/n)}{\textstyle \operatorname {E} [f(\theta _{n})-f^{*}]=O(1/n)}, where {\textstyle f^{*}}{\textstyle f^{*}} is the minimal value of {\textstyle f(\theta )}{\textstyle f(\theta )} over {\textstyle \theta \in \Theta }{\textstyle \theta \in \Theta }.[5][6]
%Conversely, in the general convex case, where we lack both the assumption of smoothness and strong convexity, Nemirovski and Yudin[7] have shown that the asymptotically optimal convergence rate, with respect to the objective function values, is {\textstyle O(1/{\sqrt {n}})}{\textstyle O(1/{\sqrt {n}})}. They have also proven that this rate cannot be improved.

\end{frame}

\begin{frame}
\frametitle{Mini-batch method}

Replace $\nabla_x Y(x,\xi_i)$ by
$$
\frac{1}{n_k} \sum_{i = 1}^{n_k} \nabla_x Y(x,\xi_i).
$$

\mbox{}

At each iteration, we take $n_k$ new draws.

\mbox{}

The cost per iteration is $n_k$ times bigger, but
\begin{itemize}
	\item it is a better estimate of the gradient
	\item the computation of the mini-batch can exploit parallelism
\end{itemize}

\end{frame}

\begin{frame}
\frametitle{Batch method}

Assume for now that the support $\Xi$ is finite, of cardinality $n$. Then
$$
\EE[Y(x,\xi)] = \frac{1}{n} \sum_{i = 1}^n Y(x,\xi_i), \quad \EE[\nabla_x Y(x,\xi)] = \frac{1}{n} \sum_{i = 1}^n \nabla_x Y(x,\xi_i)
$$

\mbox{}

Batch method:
$$
x_{k+1} = x_k - \alpha_k \frac{1}{n} \sum_{i = 1}^{n} \nabla_x Y(x,\xi_i).
$$
In other words, we use all the observations to compute the true gradient.

\mbox{}

Often, $n$ is very large, and we prefer to work with $n_k << n$.

\end{frame}

\begin{frame}
\frametitle{Stochastic approximation}

We can generalize the expression of the stochastic approximation iteration using an estimator of the gradient of $f$ at $x_k$:
$$
x_{k+1} = x_k - \alpha_k\nabla\hat{f}(x_k).
$$
As before, the gradient estimator can usually be taken as $\nabla Y ( x_k, \xi_k )$,
where $(\xi_k,\ k \geq 1)$ are i.i.d.

\mbox{}

In that case, if $f(\cdot)$ is smooth, has a unique global minimizer $x^*$, and $\alpha_k = \alpha / k$ with $\alpha > 0$ sufficiently large, then under additional nonrestrictive conditions,
$$
\sqrt{n} \left(x_n - x^*\right) \Rightarrow N(0, \Lambda), 
$$
as $n \rightarrow \infty$, for a certain $d \times d$ matrix $\Lambda$.

\end{frame}

\begin{frame}
\frametitle{Convergence speed: stochastic boundedness}

Source: \url{https://en.wikipedia.org/wiki/Big_O_in_probability_notation}

\mbox{}

We would like to measure how fast we converge to the solution, knowning that we generate a sequence of realizations of random variables.

\mbox{}

\textcolor{red}{Stochastic boundedness}

The notation
$$
X_{n}=O_{p}(a_{n}),
$$
means that the set of values $X_n/a_n$ is stochastically bounded:
$$
\forall\, \epsilon > 0,\ \exists\, M > 0, N > 0 \text{ such that } P[|X_{n}/a_{n}|>M ] < \epsilon \ \forall\, n > N.
$$

\end{frame}

\begin{frame}
\frametitle{$O_p(\cdot)$ vs $o_p(\cdot)$}

Thus, $X_n = O_p(1)$ iff
$$
\forall\, \epsilon > 0,\ \exists N_{\epsilon}, \delta_{\epsilon} \text{ such that } P[|X_{n} | \geq \delta_{\epsilon }] \leq \epsilon \ \forall\, n > N_{\epsilon}.
$$

\mbox{}

\textcolor{blue}{Convergence in probability}

$X_n = o_p(1)$ iff
$$
\forall\, \epsilon > 0, \delta > 0 \ \exists N_{\epsilon ,\delta } \text{ such that } P[|X_{n}|\geq \delta] \leq \epsilon \ \forall n > N_{\epsilon, \delta}.
$$

\mbox{}

Therefore
$$
X_n = o_p(1) \Rightarrow X_n = O_p(1).
$$
The reverse does not hold.

\mbox{}

More generally, $X_n = o_p(a_n)$ iff $X_n/{a_n} = o_p(1)$, i.e.
$$
\forall\, \epsilon > 0\ \lim_{n \rightarrow \infty} P[ |X_{n}/{a_n}| \geq \epsilon ] = 0.
$$

\end{frame}

\begin{frame}
\frametitle{Complexity}

Source: Kim, Pasupathy, and Henderson, ``A Guide to Sample Average Approximation'', in ``Handbook of Simulation Optimization'', edited by Michael C. Fu, Springer, 2015.

\mbox{}

If the number of iterations of completed in $c$ units of computer time, $n(c)$ grows roughly linearly in $c$
(as would be the case if, e.g., sample gradients are computed in constant time).

\mbox{}

A time-changed version of the CLT establishes that the resulting SA estimator has
an error
$$
x_{n(c)} - x^* = O_p(c^{-1/2}).
$$
Equivalently, the computational effort required to obtain an error of order $\epsilon$ with SA is $O_p(\epsilon^{-2})$.

\mbox{}

The performance of the recursion is highly dependent on the gain sequence $\{ \alpha_n \}$.

\end{frame}

\begin{frame}
\frametitle{Polyak–Ruppert averaging}

Within the context of the SA iterative scheme, the fastest achievable convergence rate is $O_p(c^{-1/2})$.

\mbox{}

This rate can be achieved under the ``Polyak–Ruppert averaging''.
\begin{itemize}
	\item 
step-size sequence: $a_n = a/{n^{\gamma}}$ for some $\gamma \in (0,1)$
\item
estimator of ${x}^*$:
$$
\overline{x}_n = \frac{1}{n} \sum_{i = 1}^n x_i.
$$
\end{itemize}

Under mild conditions, the Polyak–Ruppert averaging scheme enjoys a CLT, although with a different covariance matrix $\Lambda$.

\mbox{}

This happens irrespective of the value of the constant $a > 0$ (but the choice of $a$ affects the small-sample performance).
The Polyak–Ruppert averaging scheme also has other optimality properties related to
the matrix $\Lambda$.

\end{frame}

\begin{frame}
\frametitle{Order of Convergence}

Denote the numerical procedure acting on the sample function $f_n(x)$ by the mapping
$A(x) : \Theta \rightarrow \Theta$.

\mbox{}

Let $A_k(x)$ represent the iterate obtained after $k$ successive applications
of the $A(\cdot)$ on the initial iterate $x$.

\mbox{}

Assume that the function $f_n(x)$ attains its infimum $v_n^{*} := \inf\{ f_n(x) : x \in \Theta\}$ and
that $f_n (A_k (x)) \rightarrow v_n^{*}$ as $k \rightarrow \infty$ for all $x \in \Theta$.
Also, to avoid trivialities, assume that
$f_n (A_{k+1} (x))$ is different from $v_n^*$ for all $k$.

\end{frame}

\begin{frame}
\frametitle{Sublinear convergence}

Denote
$$
Q_t = \lim \sup_{k \rightarrow \infty} \frac{ | f_n(A_{k+1} (x)) - v_n^* |}
{| f_n (A_k (x)) - v_n^* |^t}.
$$

\mbox{}

\begin{definition}
$A(x) : \Theta \rightarrow \Theta$ is said to exhibit $p^{th}$-order
sublinear convergence if $Q_1 \geq 1$, and
$$
\exists\,p, s > 0 \text{ such that } p = sup \{r : f_n(A_k (x)) - v^*_n \leq s/k^r,\ \forall\, x \in \Theta\}.
$$
\end{definition}

\end{frame}

\begin{frame}
\frametitle{Linear convergence}

\begin{definition}
The numerical procedure $A(x) : \Theta \rightarrow \Theta$ is said to exhibit linear convergence if $Q_1 \in (0, 1)$ for all $x \in \Theta$.
\end{definition}

\mbox{}

The definition of linear convergence implies that there exists a constant $\beta$ satisfying $f_n ( A (x)) - v_n^* \leq \beta (f_n(x) - v_n^*)$ for all $x \in \Theta$.
The projected gradient method with Armijo steps when executed on certain smooth problems exhibits a linear convergence rate.

\end{frame}

\begin{frame}
\frametitle{Superlinear convergence}

\begin{definition}
The numerical procedure $A(x) : \Theta \rightarrow \Theta$ is said to exhibit superlinear convergence if $Q_1 = 0$ for all $x \in \Theta$.
The convergence is said to be $p^{th}$-order superlinear if $Q_1 = 0$ and $\sup \{ t : Q_t = 0 \} = p < \infty$ for all $x \in \Theta$.
\end{definition}

\mbox{}

When $f_n(x)$ is strongly convex and twice Lipschitz continuously differentiable with observable derivatives, Newton method is second-order superlinear.
For settings where the derivative is unobservable, there is a slight degradation in the convergence rate, but Newton method remains superlinear.

\end{frame}

\begin{frame}
\frametitle{Convergence rate for the SAA method}

\begin{theo}
Assumptions:
\begin{enumerate}
\item
$E[ Y^2 ( x , \xi)] < \infty$ for all $x \in \Theta$.
\item
The function $Y(x, \xi)$ is Lipschitz w.p.1, with Lipschitz constant $K(\xi)$, and $\EE[K(\xi)] < \infty$.
\item
The function $f_n(x)$ attains its infimum on $\Theta$ for each $n$ w.p.1.
\end{enumerate}
Let $c = n \times k$ and $n/c^{1 /( 2p + 1 )} \rightarrow a$ as $c \rightarrow \infty$, with $a \in (0, \infty)$.
Then, if the numerical procedure exhibits $p^{th}$-order sublinear convergence,
$$
c^{p /( 2p + 1 )} \left( f_n ( A^k ( x )) - v^* \right) = O_p(1)
$$
as $c \rightarrow \infty$.
\end{theo}

\end{frame}

\begin{frame}
\frametitle{Convergence rate for the SAA method}

\textcolor{red}{Main insight}: the maximum achievable convergence rate with the SAA method, is $O_p ( c^{- p /( 2p + 1 )})$ when the numerical
procedure in use exhibits $p^{th}$-order sublinear convergence.

\mbox{}

It is also possible to show that the corresponding rates when using linearly convergent and $p^{th}$-order superlinearly convergent procedures are $O_p (( c / \log c )^{-1/2} )$ and $O_p (( c / \log\log c )^{-1/2} )$, respectively.

\mbox{}

None of the families of numerical procedures considered are capable
of attaining the canonical convergence rate $O_p ( c^{-1/2} )$.

\end{frame}

\begin{frame}
\frametitle{The generic SG method}

Source:
\begin{itemize}
\item 
Léon Bottou, Frank E. Curtis, and Jorge Nocedal, Optimization Methods for Large-Scale Machine Learning, SIAM Review 60(2), 2018, pp. 223--311, \url{https://doi.org/10.1137/16M1080173}
\item
Léon Bottou, Frank E. Curtis, and Jorge Nocedal, Optimization Methods for Machine Learning
Part II – The theory of SG, \url{https://icml.cc/Conferences/2016/tutorials/part-2.pdf}
\end{itemize}

\mbox{}

We generalize the stochastic gradient method with the update
$$
x_{k+1} = x_k - \alpha_k g(x_k, \xi_k).
$$
instead of
$$
x_{k+1} = x_k - \alpha_k \nabla_x Y(x_k, \xi_k).
$$

\end{frame}

\begin{frame}
\frametitle{The generic SG method}

The function $f: \RR^d \rightarrow \RR$ could be
$$
f(x) =
\begin{cases}
R(x) = \EE [ Y(x; \bsxi) ] & \text{ the expected risk,} \\
R_n(x) = \frac{1}{n} \sum_{\xi = 1}^n Y(x; \xi) & \text{ the empirical risk.}
\end{cases}
$$

\mbox{}

The stochastic vector could be
$$
g(x; \xi_k) =
\begin{cases}
\nabla_x Y(x_k, \xi_k) & \text{(one realization)} \\
\frac{1}{n_k} \sum_{i = 1}^{n_k} \nabla_x Y(x_k, \xi_k) & \text{(minibatch)} \\
B_k \frac{1}{n_k} \sum_{i = 1}^{n_k} \nabla_x Y(x_k, \xi_k)
& \text{(rescaled minibatch)}
\end{cases}
$$

\end{frame}

\begin{frame}
\frametitle{Stochastic processes}

While we assumme the draws $\xi_i$, $i = 1,2,\ldots$ are i.i.d., it is possible to extend the results to the situation where $\lbrace \xi_i, i = 1,2,\ldots \rbrace$ form an adapted stochastic process, where each $\xi_i$ can depend on the previous ones.

\end{frame}

\begin{frame}
\frametitle{Active learning}

\begin{itemize}
\item 
In active learning, $g(x_k; \xi_k)$ produces a multinomial
distribution on the training examples in a manner that depends on the current solution $x_k$.
\item
$\xi_k$ is then transformed to draw from this distribution.
\end{itemize}

Active learning is not covered here, but again, the results can be extended to this situation.

\end{frame}

\begin{frame}
\frametitle{Smoothness}

\begin{theorem}
Under Assumption A.1 (Lipschitz continuity), $\forall k \in \NN$, the iterates of the SG method satisfy
\begin{multline*}
\EE_{\xi_k}[f(x_{k+1})] - f(x_k)
\\ \leq
-\alpha_k \nabla f(x_k)^T\EE_{\xi_k}[g(x_k,\xi_k)]
+\frac{1}{2} \alpha_k^2 L \EE_{\xi_k} \left[ \| g(x_k, \xi_k )\|_2^2 \right].
\end{multline*}
\end{theorem}
\begin{itemize}
	\item $\alpha_k \nabla f(x_k)^T\EE_{\xi_k}[g(x_k,\xi_k)]$: \textcolor{blue}{expected decrease};
	\item $\frac{1}{2} \alpha_k^2 L \EE_{\xi_k} \left[ \| g(x_k, \xi_k )\|_2^2 \right]$: \textcolor{blue}{noise}.
\end{itemize}

\end{frame}

\begin{frame}
	\frametitle{Smoothness: proof}
	
	From A.1, we have
	$$
	f(x_{x+1}) - f(x_k) \leq \nabla f(x_k)^T (x_{k+1} - x_k) + \frac{L}{2} \| x_{k+1} - x_k \|^2_2.
	$$
	Since $x_{k+1} = x_k - \alpha_k g(x_k, \xi_k)$, this leads to
	$$
	f(x_{x+1}) - f(x_k) \leq - \alpha_k \nabla f(x_k)^T g(x_k, \xi_k) + \alpha_k^2 \frac{L}{2} \| g(x_k, \xi_k) \|^2_2.
	$$
	This implies
    $$
		\EE_{\xi_k} [f(x_{x+1}) - f(x_k)] \leq \EE_{\xi_k} \left[ - \alpha_k \nabla f(x_k)^T g(x_k, \xi_k) + \alpha_k^2 \frac{L}{2} \| g(x_k, \xi_k) \|^2_2 \right]
	$$
	or
	$$
		\EE_{\xi_k} [f(x_{x+1})] - f(x_k) \leq - \alpha_k \nabla f(x_k)^T \EE_{\xi_k} [ g(x_k, \xi_k) ] + \alpha_k^2 \frac{L}{2} \EE_{\xi_k} \left[ \| g(x_k, \xi_k) \|^2_2 \right].	
	$$
	
	\begin{align*}
	\end{align*}
	
\end{frame}

\begin{frame}
\frametitle{Assumption A.5: first and second moment limits}

The SG method applied to $f(\cdot)$ satisfies
\begin{enumerate}[a)]
\item
The sequence of iterates $\{ x_k \}$ is contained in an open set over which $f \geq f_{lb}$.
\item
$\exists \mu, \mu_G$ such that $0 < \mu < \mu_G$ and $\forall k \in \NN$,
\begin{align*}
\nabla f(x_k)^T \EE_{\xi_k} [g(x_k, \xi_k)] &\geq \mu \| \nabla f(x_k) \|_2^2, \\
\| \EE_{\xi_k} [g(x_k, \xi_k)] \|_2 &\leq \mu_G \| \nabla f(x_k) \|_2.
\end{align*}
\item
$\exists M \geq 0, M_V \geq 0$ such that $\forall k \in \NN$,
\begin{align*}
\var_{\xi_k} [g(x_k, \xi_k)]
&= \EE_{\xi_k} \left[ \| g(x_k, \xi_k) \|^2_2 \right] - \left( \EE_{\xi_k} [\| g(x_k, \xi_k) \|_2] \right)^2 \\
&\leq M + M_V \| \nabla f(x_k) \|_2^2.
\end{align*}
\end{enumerate}
	
\end{frame}

\begin{frame}
\frametitle{Assumption A.5: notes}

\begin{itemize}
\item
A.5 b) expresses that in expectation, $g(x_k, \xi_k)$ is a sufficient descent direction.
\begin{itemize}
	\item True if $\EE_{\xi_k} [g(x_k, \xi_k)] = H_k \nabla f(x_k)$ with $H_k$ positive definite and bounded spectrum.
	\item Particular case: $H_k = I$. Then A.5 b) holds with $\mu = \mu_G = 1$. 
\end{itemize}
\item 
A.5 c) is a direct generalization of A.3.
\item
From A.5 b) and A.5 c),
\begin{align*}
\EE_{\xi_k} \left[ \| g(x_k, \xi_k) \|^2_2 \right]
& \leq \left( \EE_{\xi_k} [\| g(x_k, \xi_k) \|_2] \right)^2 + M + M_V \| \nabla f(x_k) \|_2^2 \\
& \leq \mu_G^2 \| \nabla f(x_k) \|_2^2 + M + M_V \| \nabla f(x_k) \|_2^2 \\
& = M + M_G \| \nabla f(x_k) \|_2^2,
\end{align*}
with $M_G = M_V + \mu_G^2 \geq \mu^2 > 0$.
\end{itemize}
\end{frame}

\begin{frame}
\frametitle{Moments}

\begin{theorem}
	Under Assumptions A.1 and A.5, $\forall k \in \NN$,
	\begin{align*}
		\EE_{\xi_k}[f(x_{k+1})] - f(x_k)
		& \leq - \mu \alpha_k \| \nabla f(x_k) \|^2
		+ \frac{1}{2} \alpha_k^2 L \EE_{\xi_k} \left[ \| g(x_k, \xi_k )\|_2^2 \right] \\
		& \leq - \alpha_k\left( \mu -\frac{1}{2}\alpha_k LM_G \right) \| \nabla f(x_k) \|_2^2
		+ \frac{1}{2} \alpha_k^2 LM.
	\end{align*}
\end{theorem}
\begin{itemize}
	\item $\left( \mu -\frac{1}{2}\alpha_k LM_G \right) \| \nabla f(x_k) \|_2^2$: \textcolor{blue}{expected decrease};
	\item $\frac{1}{2} \alpha_k^2 LM$: \textcolor{blue}{noise}.
\end{itemize}
\end{frame}

\begin{frame}
\frametitle{Proof}

We already proved
$$
	\EE_{\xi_k} [f(x_{x+1})] - f(x_k) \leq - \alpha_k \nabla f(x_k)^T \EE_{\xi_k} [ g(x_k, \xi_k) ] + \alpha_k^2 \frac{L}{2} \EE_{\xi_k} \left[ \| g(x_k, \xi_k) \|^2_2 \right].	
$$
From A.5 b), this leads to
$$
\EE_{\xi_k} [f(x_{x+1})] - f(x_k) \leq - \alpha_k \mu \| \nabla f(x_k) \|_2^2 + \alpha_k^2 \frac{L}{2} \EE_{\xi_k} \left[ \| g(x_k, \xi_k) \|^2_2 \right],
$$
giving the first inequality. Since
$$
\EE_{\xi_k} \left[ \| g(x_k, \xi_k) \|^2_2 \right] \leq M + M_G \| \nabla f(x_k) \|_2^2,
$$
we have
\begin{align*}
\EE_{\xi_k} [f(x_{x+1})] - f(x_k) & \leq - \alpha_k \mu \| \nabla f(x_k) \|_2^2 + \alpha_k^2 \frac{L}{2} \left( M + M_G \| \nabla f(x_k) \|_2^2 \right) \\
& =  - \alpha_k \left( \mu - \frac{1}{2} \alpha_k L M_G \right) \| \nabla f(x_k) \|_2^2 + \frac{1}{2} \alpha_k^2 LM.
\end{align*}

\end{frame}

\begin{frame}
\frametitle{Strong convexity}

Consider $f(\cdot) \in C^2$ with a convex domain.
\begin{itemize}
	\item Convexity:
	$$
	\forall x, y \in dom(f),\ f(y) \geq f(x) + \nabla f(x)^T(y-x)
	$$
	\item Strong convexity
	$$
	\exists c > 0, \forall x, y \in dom(f),\ f(y) \geq f(x) + \nabla f(x)^T(y-x) + \frac{1}{2} c \| y - x \|^2_2.
	$$
	Alternate definition if $f(\cdot) \in C^2$:
	$$
	\nabla^2 f(x) \succeq cI,
	$$
	where $I$ is the identity matrix.
\end{itemize}

Implications of strong convexity:
\begin{itemize}
	\item
	$\exists \mu > 0, \forall x, y \in dom(f),\ \| \nabla_x f(x) - \nabla_x f(y) \|_2 \geq \mu \| x - y \|_2$
	\item
	$\forall x \in dom(f), 2c (f(x) - f^*) \leq \| \nabla f(x) \|_2^2$
\end{itemize}

\end{frame}

\begin{frame}
\frametitle{Strong convexity}

\begin{itemize}
	\item Simplify the theoretical analysis and provide the strongest results.
	\item We can use regularization to locally convexify the function.
	\item Strong local minimizer:
        $\exists \epsilon > 0$ such that $f(x) > f(x^*) + \epsilon^*\| x-x^* \|_2,\ \forall\ x \in \calB(x^*,\epsilon)\setminus\{x^*\}$.\\
       Therefore, a strong local minimizer is a strict local minimizer: $\exists \epsilon > 0$ such that $f(x) > f(x^*),\ \forall\ x \in \calB(x^*,\epsilon)\setminus\{x^*\}$.
	\item If $f(x)$ is locally strongly convex in $\calB(x^*)$, any local minimizer $x^*$ is a strong local minimizer.
	\item But a strong local minimizer does not imply local strong convexity. Exemple: $f(x) = x^4$.
\end{itemize}

\end{frame}

\begin{frame}
\frametitle{Analysis}

More details at:
\begin{itemize}
\item 
\url{https://icml.cc/2016/tutorials/part-2.pdf}
\item 
\url{https://icml.cc/2016/tutorials/part-3.pdf}
\end{itemize}

We will refer to this material.

\end{frame}


\end{document}