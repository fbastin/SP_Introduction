\begin{frame}
\frametitle{First-order optimality conditions}

We denote the feasible set of a mathematical program by $\mathcal{A}$.
We will suppose here that $\mathcal{A}$ is closed.

\mbox{}

{\red Necessary condition}: if $x^*$ is a local minimizer,
\[
-\nabla_x f(x^*) \in \mathcal{N}_{\mathcal{A}}(x^*),
\]
where $\mathcal{N}_{\mathcal{A}}(x^*)$ is the normal cone to $\mathcal{A}$ at $x^*$.

\mbox{}

{\blue Cone}

\mbox
 
A subset $C$ of a vectorial space $V$ is a (linear) {\red cone}
iff $\alpha x$ belongs to $C$ for any $x \in C$ and any strictly positive scalar $\alpha$.
It is pointed if $\alpha$ can be equal to zero.

\mbox{}

A {\red convex cone} is a cone closed under convex combinations, i.e. iff $\alpha x + \beta y \in C$, $\forall \alpha,\ \beta > 0$, with $\alpha + \beta = 1$.

\end{frame}


\begin{frame}
\frametitle{Tangent cone, normal cone: general framework}

We first say that a vector $w \in \rit^n$ is tangent to $\mathcal{A}$ at $x \in \mathcal{A}$ if for all sequences of vectors $\lbrace x_i \rbrace$ with $x_i \rightarrow x$, and $x_i \in \mathcal{A}$, and all sequences of positive scalars $t_i \downarrow 0$, there exists a sequence $w_i \rightarrow w$ such that $x_i + t_iw_i \in \mathcal{A}$ for all $i$.

\mbox{}

The {\red tangent cone $T_{\mathcal{A}}(x)$} is the collection of vectors tangent to ${\mathcal{A}}$ at $x$. \\
The {\red normal cone $N_{\mathcal{A}}(x)$} is the orthogonal complement, i.e.
\[
N_{\mathcal{A}}(x) = \lbrace v\ |\ v^Tw \leq 0,\ \forall w \in T_{\mathcal{A}}(x) \rbrace.
\]

\mbox{}

OK, but in practice???

\end{frame}

\begin{frame}
\frametitle{Tangent cone, normal cone (cont'd)}

The definition of the normal cone can be simplified if ${\mathcal{A}}$ is convex.
We indeed have
\[
N_{\mathcal{A}}(x) = \left\lbrace v\ |\ v^T(x-x_0) \geq 0,\ \forall x_0 \in
{\mathcal{A}} \right\rbrace.
\]

\mbox{}

The necessary optimality condition then becomes
$\nabla_x f(\hat{x}) (x-\hat{x}) \geq 0$, $\forall x \in {\mathcal{A}}$.

\mbox{}

If $f$ is convex (i.e. we are in the convex programming framework), the condition is also sufficient.

\end{frame}

\begin{frame}
\frametitle{Deterministic and convex constraints}

Therefore, when $S$ is convex, we can rewrite the first-order criticality conditions at some point $z^*$ as the requirement that $-\nabla_z g(z^*)$ belongs to the normal cone to $S$ at $z^*$, denoted by $\mathcal{N}_{S}(z^*)$.

\mbox{}

If moreover $S$ is deterministic, the feasible sets are the same for the true and SAA problems.

\mbox{}

The previous theorem allows us to easily establish the first-order convergence.
Consider the choice $\Gamma(\cdot) = \mathcal{N}_S(\cdot)$; $\phi(z^*)$ belongs to $\Gamma(z^*)$ iff
\[
\langle \phi(z^*), u-z^*\rangle \leq 0,\ \forall u \in S.
\]

We design such variational inequalities as stochastic variational inequalities.

\end{frame}

\begin{frame}
	\frametitle{On the LICQ}
	
	This constraints qualification will be especially useful for our discusssion.
	
	\mbox{}
	
	Consider the original program.
	Recall that the active set $\mathcal{A}(z)$ at any feasible point $z$ is the union of the index set of equality constraints and active inequality constraints:
	\[
	\mathcal{A}(z) = \lbrace i \in \lbrace 1,\ldots,k \rbrace \, | \,
	c_i(z) = 0 \rbrace \cup \lbrace k+1,\ldots,M \rbrace.
	\]
	
	\begin{defi}
		Given the point $z^*$ and the active set $\mathcal{A}(z^*)$, we say that the LICQ holds at $z^*$ if the set of active constraints gradients $\lbrace \nabla c_j(z^*),\ j \in \mathcal{A}(z^*) \rbrace$ is linearly independent.
	\end{defi}
	
\end{frame}

\begin{frame}
	\frametitle{Strict complementarity}
	
	Another useful concept for our needs is the strict complementarity condition.
	
	\begin{defi}
		Given $z^*$ and a vector $\lambda^*$ satisfying the KKT conditions, we say that the strict complementarity condition holds if exactly one of $[\lambda]_j^*$ and $c_j(z^*)$ is null for every index $j=1,\ldots,k$, i.e. we have that $[\lambda]_j^* > 0$ for each $j \in \lbrace 1,\ldots,k \rbrace \cap \mathcal{A}(z^*)$.
	\end{defi}
	
	\mbox{}
	
	Consider again a particular sampling process $\overline{\xi}$ dans $(\Xi_{\Pi}, \mathcal{F}_{\Pi}, P_{\Pi})$, such that $\epsilon_N (z, \overline{\xi} ) \rightarrow 0$ as $N \rightarrow \infty$. 
	If the assumptions of the above lemma hold for some subsequence $\lbrace z_{\ell}^* \rbrace_{\ell = 1}^{\infty} \rightarrow z^*$, the gradient of the Lagrangian $\nabla L\left( z^*_{\ell} ( \overline{\xi} ), \lambda_{\ell}^* ( \overline{\xi} ) \right)$ converges to $\nabla \mathcal{L} ( z^*, \lambda^* )$, as $\ell$ tends to infinity, with $\lambda_{\ell}^*(\overline{\xi}) \rightarrow \lambda^*$.
	
\end{frame}

\begin{frame}
	\frametitle{Strict complementarity (cont'd)}
	
	Assume that the strict complementarity condition holds at $z^*$ for the approximate problem.
	We obtain that for $\ell$ large enough,
	$[\lambda^*_{\ell}]_j(\overline{\xi})$, $j \in \lbrace 1,\ldots,k
	\rbrace \cap \mathcal{A} ( z^* ) $, are strictly positive and thus the corresponding constraints are active at $z^*_{\ell}(\overline{\xi})$.
	
	\mbox{}
	
	Moreover, as $\epsilon_N (z, \overline{\xi} ) \rightarrow 0$,
	$\hat{c}_j \left( z^*_{\ell}(\overline{\xi}),
	\epsilon_{\ell}(z^*_{\ell}(\overline{\xi}),\overline{\xi}) \right) \rightarrow
	c_j ( z^* )$ and, for $\ell$ large enough,
	%(which is true for almost every  $\overline{\xi}$)
	$\mathcal{A} \left( z^*_{\ell}(\overline{\xi}) \right) = \mathcal{A} (
	z^* )$, so the strict complementarity condition holds at $\left(
	z^*_{\ell}(\overline{\xi}), \lambda_{\ell}^*(\overline{\xi})
	\right)$ for the approximate problem.
	
	\mbox{}
	
	This allows us to state a second-order convergence result.
	
\end{frame}

