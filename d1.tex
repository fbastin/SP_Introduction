\begin{frame}
\frametitle{Clear enough?}

If we consider the farmer problem example, is the recourse complete, relatively complete, simple?

\end{frame}

begin{frame}
\frametitle{Exercice 2}

%(Birge et Louveaux, page 102)
Soit le programme de seconde étape défini par
\begin{align*}
\min_y\ & 2y_1+y_2\\
\mbox{t.q. } & y_1 + y_2 \geq 1-x_1,\\
& y_1 \geq \bxi - x_1-x_2,\\
& y_1, y_2 \geq 0.
\end{align*}
Montrez que ce programme a un recours complet si $\bxi$ est
d'espérance finie.

\mbox{}

Recours complet? Il faut que pour tout $z \in \rit^{2}$, on puisse
trouver $y \geq 0$
t.q. $Wy = z$. Le recours complet est une propriété de $W$, mais vaut
$W$? De manière équivalente, on veut pos$W = \rit^2$.

\end{frame}

\begin{frame}
	\frametitle{Exercice 4 - suite}
	
	Petite difficulté: le problème n'est pas sous forme standard.
	L'idée est en fait de montrer pour pour tout $z \in \rit^2$, il existe
	$y \geq 0$ t.q $Wy \geq z$. L'équivalence avec pos$W$ ne fonctionne
	plus. Sauf si nous nous ramenons à la forme standard!
	
	\mbox{}
	
	Sous forme standard, le problème devient
	\begin{align*}
	\min_y\ & 2y_1+y_2\\
	\mbox{t.q. } & y_1 + y_2 - s_1 = 1-x_1,\\
	& y_1 - s_2 = \bxi - x_1-x_2,\\
	& y_1, y_2 \geq 0,\\
	& s_1, s_2 \geq 0,
	\end{align*}
	et il y a recours complet si pour tout $z \in R^2$, il existe $y$, $s
	\geq 0$ vérifiant
	\[
	Wy = z.
	\]
	
\end{frame}

\begin{frame}
	\frametitle{Exercice 2 - suite}
	
	Explicitons $Wy = z$:
	\[
	\begin{pmatrix}
	1 & 1 & -1 & 0 \\
	1 & 0 & 0  & -1
	\end{pmatrix}
	\begin{pmatrix} y_1 \\ y_2 \\ s_1 \\ s_2 \end{pmatrix} =
	\begin{pmatrix} z_1 \\ z_2 \end{pmatrix}.
	\]
	Autrement dit, nous devons trouver $y_1$, $y_2$, $s_1$, $s_2 \geq 0$
	tel que
	\begin{align*}
	y_1 + y_2 - s_1 &= z_1. \\
	y_1 - s_2 &= z_2.
	\end{align*}
	
	La deuxième équation nous donne la possibilité de prendre
	\[
	\begin{cases}
	y_1 = z_2,\ s_2 = 0 & \mbox{si } z_2 \geq 0,\\
	y_1 = 0,\ s_2 = -z_2 & \mbox{sinon}.\\
	\end{cases}
	\]
	
\end{frame}

\begin{frame}
	\frametitle{Exercice 2 - suite}
	
	Nous avons aussi
	\[
	y_2 - s_1 = z_1 - y_1.
	\]
	En utilisant nos précédents choix, nous pouvons prendre pour $y_2$ ce
	qui suit. Si $z_2 \geq 0$, $y_2 - s_1 = z_1 - z_2$, et nous prenons
	\[
	\begin{cases}
	y_2 = z_1-z_2,\ s_1 = 0 & \mbox{si } z_1-z_2 \geq 0,\\
	y_2 = 0,\ s_2 = z_2-z_1 & \mbox{sinon}.\\
	\end{cases}
	\]
	De même, si $z_2 \leq 0$, $y_2 - s_1 = z_1$, et il suffit de choisir
	\[
	\begin{cases}
	y_2 = z_1,\ s_1 = 0 & \mbox{si } z_1 \geq 0,\\
	y_2 = 0,\ s_2 = -z_1 & \mbox{sinon}.\\
	\end{cases}
	\]
	
\end{frame}

\begin{frame}
	\frametitle{Exercice 2 - suite}
	
	Par conséquent, quel que soit $z$, on peut trouver une solution
	non-négative au système
	\[
	W\begin{pmatrix} y \\ s \end{pmatrix} = z,
	\]
	i.e. le recours est complet!
	
	%\mbox{}
	
	%Pourquoi exiger $\bxi$ d'espérance finie?
	
	%\mbox{}
	
	%Oups,\ldots pas utile à ce stade (copie de l'erreur dans l'énoncé
	%repris de Birge et Louveaux page 102), puisque le recours complet
	%n'est une propriété que de $W$, fixe.
	La condition d'espérance finie sert surtout à garantir que le problème
	de seconde étape est bien défini.
	On peut en effet montrer que $\mathcal{Q}(x)$ n'est pas fini si
	$\bxi$ n'est pas d'espérance finie (l'expression de $\mathcal{Q}(x)$
	fait intervenir $\bxi$).
	
\end{frame}

\begin{frame}
	\frametitle{Exercice 3}
	
	Considérons le programme de deuxième défini comme suit:
	\begin{align*}
	\min_y\ & 2y_1+y_2\\
	\mbox{t.q. } & y_1 - y_2 \leq 2-\bxi x_1,\\
	& y_2 \leq x_2,\\
	& y_1, y_2 \geq 0.
	\end{align*}
	Trouvez $K_2(\bxi)$ et $K_2$ pour:
	\begin{enumerate}
		\item
		$\bxi \sim U[0,1]$.
		\item
		$\bxi \sim \mbox{Poisson}(\lambda),\ \lambda > 0$.
	\end{enumerate}
	Quelles propriétés attendez-vous pour $K_2$?
	
\end{frame}

\begin{frame}
	\frametitle{Exercice 3 - suite}
	
	Nous devons donc déterminer des ensembles réalisables.
	
	\mbox{}
	
	Remarquons tout d'abord que comme $y_2 \geq 0$ et $y_2 \leq x_2$, nous
	devons avoir $x_2 \geq 0$.
	
	\mbox{}
	
	Utilisons la contrainte
	\[
	y_1 - y_2 \leq 2-\bxi x_1.
	\]
	Comme $y_2 \leq x_2$, nous avons $y_1 \leq 2-\bxi x_1 + x_2$, et tout
	$y_1$ satisfaisant cette inégalité conviendra. Le
	problème est donc rélisable si et seulement si $2-\bxi x_1 + x_2 \geq 0$.
	
	Il suit que
	\begin{align*}
	K_2(\xi) &= \lbrace x = (x_1, x_2)^T \,|\, x_2 \geq 0, \xi x_1 \leq
	2+x_2 \rbrace \\
	&= \lbrace  x = (x_1, x_2)^T \,|\, x_2 \geq \max(0, \xi x_1 -2 ) \rbrace.
	\end{align*}
	
\end{frame}

\begin{frame}
	\frametitle{Exercice 3 - Suite}
	
	Pour les distributions considérées, $E[\xi]$ et $E[\xi^2]$ existent et
	sont finies. Il suit que
	\[
	K_2 = K_2^P = \cap_{\xi \in \Xi} K_2(\xi).
	\]
	
	\mbox{}
	
	Si $\bxi \sim U[0,1]$, $\xi \leq 1$, et $\xi x_1 \leq 2 + x_2$ si $x_1
	\leq 2 + x_2$. Autrement dit, $K_2(\xi) \subseteq K_2(1)$, et
	\[
	K_2 = \lbrace x \,|\, x_2 \geq 0, x_1 \leq 2 + x_2 \rbrace.
	\]
	
	\mbox{}
	
	Si $\bxi \sim \mbox{Poisson}(\lambda)$, $\xi = 0, 1,\ldots,$, de sorte
	que $\xi x_1$ n'est pas borné supérieurement, sauf si $x_1 \leq
	0$. Dès lors,
	\[
	K_2 = \lbrace x \,|\, x_1 \leq 0, x_2 \geq 0 \rbrace.
	\]
	
\end{frame}

\begin{frame}
	\frametitle{Exercice 3 - Suite}
	
	Interprétation. Toute décision de première étape doit satisfaire
	$x_1$, $x_2 \geq 0$, par conséquent, si $\bxi \sim U[0,1]$,
	\[
	K_1 \cap K_2 \subseteq \lbrace x \,|\, x_1, x_2 \geq 0, x_1 \leq 2+x_2
	\rbrace,
	\]
	et si $\bxi \sim \mbox{Poisson}(\lambda)$,
	\[
	K_1 \cap K_2 \subseteq \lbrace x \,|\, x_1 = 0, x_2 \geq 0 \rbrace.
	\]
	Nous pouvons espérer un recours relativement complet dans le premier
	cas (cela dépend de $K_1$), mais le deuxième se présente assez mal!
	
\end{frame}

\begin{frame}
	\frametitle{Recourse function}
	
	\begin{itemize}
		\item
		The ease of resolution depends on the properties of $\mathcal{Q}(x)$, and therefore should be studied! (See also Kall and Wallace, Section~1.4).
		\item
		If the duality equality holds,
		\[
		v(z) = \min_{y \in \rit^p} \lbrace q(\omega)^Ty \ |\ Wy = z \rbrace
		= \max_{t \in \rit^m} \lbrace z^Tt \ |\ W^Tt \leq q \rbrace.
		\]
		\item
		Let $\Lambda = \lbrace \lambda_1, \lambda_2, \ldots,
		\lambda_{|\Lambda|}\rbrace$ be the set of extremal points of $\lbrace t \in \rit^m \ |\ W^Tt \leq q\rbrace$.
		\begin{itemize}
			\item
			Each extremal point $\lambda_k$ is potentially an optimal solution of the linear program.
			\item
			By the fundamental theorem of linear programming, if there exists an optimal solution, one of the extremal point is an optimal solution, therefore
			\[
			v(z) = \max_{k=1,\ldots,|\Lambda|} z^T\lambda_k,\quad z \in \rit^m.
			\]
		\end{itemize}
	\end{itemize}
\end{frame}
