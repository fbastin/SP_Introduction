\documentclass[french]{beamer}

\usepackage[utf8]{inputenc}
\usetheme{Singapore}
\usepackage{xcolor}
\setbeamertemplate{footline}[frame number]

\usepackage{mathlist}

\usepackage[utf8]{inputenc}
\usepackage{amsmath,amssymb}

\newtheorem{defi}{Définition}
\newtheorem{theo}{Théorème}
\newtheorem{coro}{Corollaire}
\newtheorem{lem}{Lemme}

\usepackage{epic,eepic}
\usepackage{pstricks, pst-tree}

\usepackage{epstopdf}
\usepackage{auto-pst-pdf}

\usepackage{times}

\usepackage{ulem}

\usepackage{listings}
%\topmargin=-0.6in
\lstloadlanguages{C++}
\lstset{language=C++}
%%}

\usepackage{mathtools}

\setbeamercovered{dynamic}

% We add the following commonly used macros:

% vectors as boldsymbols:
\newcommand{\bsa}{{\boldsymbol{a}}}
\newcommand{\bsb}{{\boldsymbol{b}}}
\newcommand{\bsc}{{\boldsymbol{c}}}
\newcommand{\bsd}{{\boldsymbol{d}}}
\newcommand{\bse}{{\boldsymbol{e}}}
\newcommand{\bsf}{{\boldsymbol{f}}}
\newcommand{\bsg}{{\boldsymbol{g}}}
\newcommand{\bsh}{{\boldsymbol{h}}}
\newcommand{\bsi}{{\boldsymbol{i}}}
\newcommand{\bsj}{{\boldsymbol{j}}}
\newcommand{\bsk}{{\boldsymbol{k}}}
\newcommand{\bsl}{{\boldsymbol{l}}}
\newcommand{\bsm}{{\boldsymbol{m}}}
\newcommand{\bsn}{{\boldsymbol{n}}}
\newcommand{\bso}{{\boldsymbol{o}}}
\newcommand{\bsp}{{\boldsymbol{p}}}
\newcommand{\bsq}{{\boldsymbol{q}}}
\newcommand{\bsr}{{\boldsymbol{r}}}
\newcommand{\bss}{{\boldsymbol{s}}}
\newcommand{\bst}{{\boldsymbol{t}}}
\newcommand{\bsu}{{\boldsymbol{u}}}
\newcommand{\bsv}{{\boldsymbol{v}}}
\newcommand{\bsw}{{\boldsymbol{w}}}
\newcommand{\bsx}{{\boldsymbol{x}}}
\newcommand{\bsy}{{\boldsymbol{y}}}
\newcommand{\bsz}{{\boldsymbol{z}}}
\newcommand{\bsA}{{\boldsymbol{A}}}
\newcommand{\bsB}{{\boldsymbol{B}}}
\newcommand{\bsC}{{\boldsymbol{C}}}
\newcommand{\bsD}{{\boldsymbol{D}}}
\newcommand{\bsE}{{\boldsymbol{E}}}
\newcommand{\bsF}{{\boldsymbol{F}}}
\newcommand{\bsG}{{\boldsymbol{G}}}
\newcommand{\bsH}{{\boldsymbol{H}}}
\newcommand{\bsI}{{\boldsymbol{I}}}
\newcommand{\bsJ}{{\boldsymbol{J}}}
\newcommand{\bsK}{{\boldsymbol{K}}}
\newcommand{\bsL}{{\boldsymbol{L}}}
\newcommand{\bsM}{{\boldsymbol{M}}}
\newcommand{\bsN}{{\boldsymbol{N}}}
\newcommand{\bsO}{{\boldsymbol{O}}}
\newcommand{\bsP}{{\boldsymbol{P}}}
\newcommand{\bsQ}{{\boldsymbol{Q}}}
\newcommand{\bsR}{{\boldsymbol{R}}}
\newcommand{\bsS}{{\boldsymbol{S}}}
\newcommand{\bsT}{{\boldsymbol{T}}}
\newcommand{\bsU}{{\boldsymbol{U}}}
\newcommand{\bsV}{{\boldsymbol{V}}}
\newcommand{\bsW}{{\boldsymbol{W}}}
\newcommand{\bsX}{{\boldsymbol{X}}}
\newcommand{\bsY}{{\boldsymbol{Y}}}
\newcommand{\bsZ}{{\boldsymbol{Z}}}
% other commonly used boldsymbols:
\newcommand{\bsell}{{\boldsymbol{\ell}}}
\newcommand{\bszero}{{\boldsymbol{0}}} % vector of zeros
\newcommand{\bsone}{{\boldsymbol{1}}}  % vector of ones
% boldsymbol greeks:
\newcommand{\bsalpha}{{\boldsymbol{\alpha}}}
\newcommand{\bsbeta}{{\boldsymbol{\beta}}}
\newcommand{\bsgamma}{{\boldsymbol{\gamma}}}
\newcommand{\bsdelta}{{\boldsymbol{\delta}}}
\newcommand{\bsepsilon}{{\boldsymbol{\epsilon}}}
\newcommand{\bsvarepsilon}{{\boldsymbol{\varepsilon}}}
\newcommand{\bszeta}{{\boldsymbol{\zeta}}}
\newcommand{\bseta}{{\boldsymbol{\eta}}}
\newcommand{\bstheta}{{\boldsymbol{\theta}}}
\newcommand{\bsvartheta}{{\boldsymbol{\vartheta}}}
\newcommand{\bskappa}{{\boldsymbol{\kappa}}}
\newcommand{\bslambda}{{\boldsymbol{\lambda}}}
\newcommand{\bsmu}{{\boldsymbol{\mu}}}
\newcommand{\bsnu}{{\boldsymbol{\nu}}}
\newcommand{\bsxi}{{\boldsymbol{\xi}}}
\newcommand{\bspi}{{\boldsymbol{\pi}}}
\newcommand{\bsvarpi}{{\boldsymbol{\varpi}}}
\newcommand{\bsrho}{{\boldsymbol{\rho}}}
\newcommand{\bsvarrho}{{\boldsymbol{\varrho}}}
\newcommand{\bssigma}{{\boldsymbol{\sigma}}}
\newcommand{\bsvarsigma}{{\boldsymbol{\varsigma}}}
\newcommand{\bstau}{{\boldsymbol{\tau}}}
\newcommand{\bsupsilon}{{\boldsymbol{\upsilon}}}
\newcommand{\bsphi}{{\boldsymbol{\phi}}}
\newcommand{\bsvarphi}{{\boldsymbol{\varphi}}}
\newcommand{\bschi}{{\boldsymbol{\chi}}}
\newcommand{\bspsi}{{\boldsymbol{\psi}}}
\newcommand{\bsomega}{{\boldsymbol{\omega}}}
\newcommand{\bsGamma}{{\boldsymbol{\Gamma}}}
\newcommand{\bsDelta}{{\boldsymbol{\Delta}}}
\newcommand{\bsTheta}{{\boldsymbol{\Theta}}}
\newcommand{\bsLambda}{{\boldsymbol{\Lambda}}}
\newcommand{\bsXi}{{\boldsymbol{\Xi}}}
\newcommand{\bsPi}{{\boldsymbol{\Pi}}}
\newcommand{\bsSigma}{{\boldsymbol{\Sigma}}}
\newcommand{\bsUpsilon}{{\boldsymbol{\Upsilon}}}
\newcommand{\bsPhi}{{\boldsymbol{\Phi}}}
\newcommand{\bsPsi}{{\boldsymbol{\Psi}}}
\newcommand{\bsOmega}{{\boldsymbol{\Omega}}}

% Roman fonts:
\newcommand{\rma}{{\mathrm{a}}}
\newcommand{\rmb}{{\mathrm{b}}}
\newcommand{\rmc}{{\mathrm{c}}}
\newcommand{\rmd}{{\mathrm{d}}}
\newcommand{\rme}{{\mathrm{e}}}
\newcommand{\rmf}{{\mathrm{f}}}
\newcommand{\rmg}{{\mathrm{g}}}
\newcommand{\rmh}{{\mathrm{h}}}
\newcommand{\rmi}{{\mathrm{i}}}
\newcommand{\rmj}{{\mathrm{j}}}
\newcommand{\rmk}{{\mathrm{k}}}
\newcommand{\rml}{{\mathrm{l}}}
\newcommand{\rmm}{{\mathrm{m}}}
\newcommand{\rmn}{{\mathrm{n}}}
\newcommand{\rmo}{{\mathrm{o}}}
\newcommand{\rmp}{{\mathrm{p}}}
\newcommand{\rmq}{{\mathrm{q}}}
\newcommand{\rmr}{{\mathrm{r}}}
\newcommand{\rms}{{\mathrm{s}}}
\newcommand{\rmt}{{\mathrm{t}}}
\newcommand{\rmu}{{\mathrm{u}}}
\newcommand{\rmv}{{\mathrm{v}}}
\newcommand{\rmw}{{\mathrm{w}}}
\newcommand{\rmx}{{\mathrm{x}}}
\newcommand{\rmy}{{\mathrm{y}}}
\newcommand{\rmz}{{\mathrm{z}}}
\newcommand{\rmA}{{\mathrm{A}}}
\newcommand{\rmB}{{\mathrm{B}}}
\newcommand{\rmC}{{\mathrm{C}}}
\newcommand{\rmD}{{\mathrm{D}}}
\newcommand{\rmE}{{\mathrm{E}}}
\newcommand{\rmF}{{\mathrm{F}}}
\newcommand{\rmG}{{\mathrm{G}}}
\newcommand{\rmH}{{\mathrm{H}}}
\newcommand{\rmI}{{\mathrm{I}}}
\newcommand{\rmJ}{{\mathrm{J}}}
\newcommand{\rmK}{{\mathrm{K}}}
\newcommand{\rmL}{{\mathrm{L}}}
\newcommand{\rmM}{{\mathrm{M}}}
\newcommand{\rmN}{{\mathrm{N}}}
\newcommand{\rmO}{{\mathrm{O}}}
\newcommand{\rmP}{{\mathrm{P}}}
\newcommand{\rmQ}{{\mathrm{Q}}}
\newcommand{\rmR}{{\mathrm{R}}}
\newcommand{\rmS}{{\mathrm{S}}}
\newcommand{\rmT}{{\mathrm{T}}}
\newcommand{\rmU}{{\mathrm{U}}}
\newcommand{\rmV}{{\mathrm{V}}}
\newcommand{\rmW}{{\mathrm{W}}}
\newcommand{\rmX}{{\mathrm{X}}}
\newcommand{\rmY}{{\mathrm{Y}}}
\newcommand{\rmZ}{{\mathrm{Z}}}
% also commonly defined
\newcommand{\rd}{{\mathrm{d}}}
\newcommand{\ri}{{\mathrm{i}}}

% blackboards:
\newcommand{\bbA}{{\mathbb{A}}}
\newcommand{\bbB}{{\mathbb{B}}}
\newcommand{\bbC}{{\mathbb{C}}}
\newcommand{\bbD}{{\mathbb{D}}}
\newcommand{\bbE}{{\mathbb{E}}}
\newcommand{\bbF}{{\mathbb{F}}}
\newcommand{\bbG}{{\mathbb{G}}}
\newcommand{\bbH}{{\mathbb{H}}}
\newcommand{\bbI}{{\mathbb{I}}}
\newcommand{\bbJ}{{\mathbb{J}}}
\newcommand{\bbK}{{\mathbb{K}}}
\newcommand{\bbL}{{\mathbb{L}}}
\newcommand{\bbM}{{\mathbb{M}}}
\newcommand{\bbN}{{\mathbb{N}}}
\newcommand{\bbO}{{\mathbb{O}}}
\newcommand{\bbP}{{\mathbb{P}}}
\newcommand{\bbQ}{{\mathbb{Q}}}
\newcommand{\bbR}{{\mathbb{R}}}
\newcommand{\bbS}{{\mathbb{S}}}
\newcommand{\bbT}{{\mathbb{T}}}
\newcommand{\bbU}{{\mathbb{U}}}
\newcommand{\bbV}{{\mathbb{V}}}
\newcommand{\bbW}{{\mathbb{W}}}
\newcommand{\bbX}{{\mathbb{X}}}
\newcommand{\bbY}{{\mathbb{Y}}}
\newcommand{\bbZ}{{\mathbb{Z}}}
% commonly used shortcuts:
\newcommand{\C}{{\mathbb{C}}} % complex numbers
\newcommand{\F}{{\mathbb{F}}} % field, finite field
\newcommand{\N}{{\mathbb{N}}} % natural numbers {1, 2, ...}
\newcommand{\Q}{{\mathbb{Q}}} % rationals
\newcommand{\R}{{\mathbb{R}}} % reals
\newcommand{\Z}{{\mathbb{Z}}} % integers
% more commonly used shortcuts:
\newcommand{\CC}{{\mathbb{C}}} % complex numbers
\newcommand{\FF}{{\mathbb{F}}} % field, finite field
\newcommand{\NN}{{\mathbb{N}}} % natural numbers {1, 2, ...}
\newcommand{\QQ}{{\mathbb{Q}}} % rationals
\newcommand{\RR}{{\mathbb{R}}} % reals
\newcommand{\ZZ}{{\mathbb{Z}}} % integers
% more commonly used shortcuts:
\newcommand{\EE}{{\mathbb{E}}}
\newcommand{\PP}{{\mathbb{P}}}
\newcommand{\TT}{{\mathbb{T}}}
\newcommand{\VV}{{\mathbb{V}}}
% and even more commonly used shortcuts:
\newcommand{\Complex}{{\mathbb{C}}}
\newcommand{\Integer}{{\mathbb{Z}}}
\newcommand{\Natural}{{\mathbb{N}}}
\newcommand{\Rational}{{\mathbb{Q}}}
\newcommand{\Real}{{\mathbb{R}}}

%%% Issue with miktex
% indicator boldface 1:
%\DeclareSymbolFont{bbold}{U}{bbold}{m}{n}
%\DeclareSymbolFontAlphabet{\mathbbold}{bbold}
%\newcommand{\ind}{{\mathbbold{1}}}
%\newcommand{\bbone}{{\mathbbold{1}}}


% calligraphic letters:
\newcommand{\cala}{{\mathcal{a}}}
\newcommand{\calb}{{\mathcal{b}}}
\newcommand{\calc}{{\mathcal{c}}}
\newcommand{\cald}{{\mathcal{d}}}
\newcommand{\cale}{{\mathcal{e}}}
\newcommand{\calf}{{\mathcal{f}}}
\newcommand{\calg}{{\mathcal{g}}}
\newcommand{\calh}{{\mathcal{h}}}
\newcommand{\cali}{{\mathcal{i}}}
\newcommand{\calj}{{\mathcal{j}}}
\newcommand{\calk}{{\mathcal{k}}}
\newcommand{\call}{{\mathcal{l}}}
\newcommand{\calm}{{\mathcal{m}}}
\newcommand{\caln}{{\mathcal{n}}}
\newcommand{\calo}{{\mathcal{o}}}
\newcommand{\calp}{{\mathcal{p}}}
\newcommand{\calq}{{\mathcal{q}}}
\newcommand{\calr}{{\mathcal{r}}}
\newcommand{\cals}{{\mathcal{s}}}
\newcommand{\calt}{{\mathcal{t}}}
\newcommand{\calu}{{\mathcal{u}}}
\newcommand{\calv}{{\mathcal{v}}}
\newcommand{\calw}{{\mathcal{w}}}
\newcommand{\calx}{{\mathcal{x}}}
\newcommand{\caly}{{\mathcal{y}}}
\newcommand{\calz}{{\mathcal{z}}}
\newcommand{\calA}{{\mathcal{A}}}
\newcommand{\calB}{{\mathcal{B}}}
\newcommand{\calC}{{\mathcal{C}}}
\newcommand{\calD}{{\mathcal{D}}}
\newcommand{\calE}{{\mathcal{E}}}
\newcommand{\calF}{{\mathcal{F}}}
\newcommand{\calG}{{\mathcal{G}}}
\newcommand{\calH}{{\mathcal{H}}}
\newcommand{\calI}{{\mathcal{I}}}
\newcommand{\calJ}{{\mathcal{J}}}
\newcommand{\calK}{{\mathcal{K}}}
\newcommand{\calL}{{\mathcal{L}}}
\newcommand{\calM}{{\mathcal{M}}}
\newcommand{\calN}{{\mathcal{N}}}
\newcommand{\calO}{{\mathcal{O}}}
\newcommand{\calP}{{\mathcal{P}}}
\newcommand{\calQ}{{\mathcal{Q}}}
\newcommand{\calR}{{\mathcal{R}}}
\newcommand{\calS}{{\mathcal{S}}}
\newcommand{\calT}{{\mathcal{T}}}
\newcommand{\calU}{{\mathcal{U}}}
\newcommand{\calV}{{\mathcal{V}}}
\newcommand{\calW}{{\mathcal{W}}}
\newcommand{\calX}{{\mathcal{X}}}
\newcommand{\calY}{{\mathcal{Y}}}
\newcommand{\calZ}{{\mathcal{Z}}}

% Euler fraks:
\newcommand{\fraka}{{\mathfrak{a}}}
\newcommand{\frakb}{{\mathfrak{b}}}
\newcommand{\frakc}{{\mathfrak{c}}}
\newcommand{\frakd}{{\mathfrak{d}}}
\newcommand{\frake}{{\mathfrak{e}}}
\newcommand{\frakf}{{\mathfrak{f}}}
\newcommand{\frakg}{{\mathfrak{g}}}
\newcommand{\frakh}{{\mathfrak{h}}}
\newcommand{\fraki}{{\mathfrak{i}}}
\newcommand{\frakj}{{\mathfrak{j}}}
\newcommand{\frakk}{{\mathfrak{k}}}
\newcommand{\frakl}{{\mathfrak{l}}}
\newcommand{\frakm}{{\mathfrak{m}}}
\newcommand{\frakn}{{\mathfrak{n}}}
\newcommand{\frako}{{\mathfrak{o}}}
\newcommand{\frakp}{{\mathfrak{p}}}
\newcommand{\frakq}{{\mathfrak{q}}}
\newcommand{\frakr}{{\mathfrak{r}}}
\newcommand{\fraks}{{\mathfrak{s}}}
\newcommand{\frakt}{{\mathfrak{t}}}
\newcommand{\fraku}{{\mathfrak{u}}}
\newcommand{\frakv}{{\mathfrak{v}}}
\newcommand{\frakw}{{\mathfrak{w}}}
\newcommand{\frakx}{{\mathfrak{x}}}
\newcommand{\fraky}{{\mathfrak{y}}}
\newcommand{\frakz}{{\mathfrak{z}}}
\newcommand{\frakA}{{\mathfrak{A}}}
\newcommand{\frakB}{{\mathfrak{B}}}
\newcommand{\frakC}{{\mathfrak{C}}}
\newcommand{\frakD}{{\mathfrak{D}}}
\newcommand{\frakE}{{\mathfrak{E}}}
\newcommand{\frakF}{{\mathfrak{F}}}
\newcommand{\frakG}{{\mathfrak{G}}}
\newcommand{\frakH}{{\mathfrak{H}}}
\newcommand{\frakI}{{\mathfrak{I}}}
\newcommand{\frakJ}{{\mathfrak{J}}}
\newcommand{\frakK}{{\mathfrak{K}}}
\newcommand{\frakL}{{\mathfrak{L}}}
\newcommand{\frakM}{{\mathfrak{M}}}
\newcommand{\frakN}{{\mathfrak{N}}}
\newcommand{\frakO}{{\mathfrak{O}}}
\newcommand{\frakP}{{\mathfrak{P}}}
\newcommand{\frakQ}{{\mathfrak{Q}}}
\newcommand{\frakR}{{\mathfrak{R}}}
\newcommand{\frakS}{{\mathfrak{S}}}
\newcommand{\frakT}{{\mathfrak{T}}}
\newcommand{\frakU}{{\mathfrak{U}}}
\newcommand{\frakV}{{\mathfrak{V}}}
\newcommand{\frakW}{{\mathfrak{W}}}
\newcommand{\frakX}{{\mathfrak{X}}}
\newcommand{\frakY}{{\mathfrak{Y}}}
\newcommand{\frakZ}{{\mathfrak{Z}}}
% sets as Euler fraks:
\newcommand{\seta}{{\mathfrak{a}}}
\newcommand{\setb}{{\mathfrak{b}}}
\newcommand{\setc}{{\mathfrak{c}}}
\newcommand{\setd}{{\mathfrak{d}}}
\newcommand{\sete}{{\mathfrak{e}}}
\newcommand{\setf}{{\mathfrak{f}}}
\newcommand{\setg}{{\mathfrak{g}}}
\newcommand{\seth}{{\mathfrak{h}}}
\newcommand{\seti}{{\mathfrak{i}}}
\newcommand{\setj}{{\mathfrak{j}}}
\newcommand{\setk}{{\mathfrak{k}}}
\newcommand{\setl}{{\mathfrak{l}}}
\newcommand{\setm}{{\mathfrak{m}}}
\newcommand{\setn}{{\mathfrak{n}}}
\newcommand{\seto}{{\mathfrak{o}}}
\newcommand{\setp}{{\mathfrak{p}}}
\newcommand{\setq}{{\mathfrak{q}}}
\newcommand{\setr}{{\mathfrak{r}}}
\newcommand{\sets}{{\mathfrak{s}}}
\newcommand{\sett}{{\mathfrak{t}}}
\newcommand{\setu}{{\mathfrak{u}}}
\newcommand{\setv}{{\mathfrak{v}}}
\newcommand{\setw}{{\mathfrak{w}}}
\newcommand{\setx}{{\mathfrak{x}}}
\newcommand{\sety}{{\mathfrak{y}}}
\newcommand{\setz}{{\mathfrak{z}}}
\newcommand{\setA}{{\mathfrak{A}}}
\newcommand{\setB}{{\mathfrak{B}}}
\newcommand{\setC}{{\mathfrak{C}}}
\newcommand{\setD}{{\mathfrak{D}}}
\newcommand{\setE}{{\mathfrak{E}}}
\newcommand{\setF}{{\mathfrak{F}}}
\newcommand{\setG}{{\mathfrak{G}}}
\newcommand{\setH}{{\mathfrak{H}}}
\newcommand{\setI}{{\mathfrak{I}}}
\newcommand{\setJ}{{\mathfrak{J}}}
\newcommand{\setK}{{\mathfrak{K}}}
\newcommand{\setL}{{\mathfrak{L}}}
\newcommand{\setM}{{\mathfrak{M}}}
\newcommand{\setN}{{\mathfrak{N}}}
\newcommand{\setO}{{\mathfrak{O}}}
\newcommand{\setP}{{\mathfrak{P}}}
\newcommand{\setQ}{{\mathfrak{Q}}}
\newcommand{\setR}{{\mathfrak{R}}}
\newcommand{\setS}{{\mathfrak{S}}}
\newcommand{\setT}{{\mathfrak{T}}}
\newcommand{\setU}{{\mathfrak{U}}}
\newcommand{\setV}{{\mathfrak{V}}}
\newcommand{\setW}{{\mathfrak{W}}}
\newcommand{\setX}{{\mathfrak{X}}}
\newcommand{\setY}{{\mathfrak{Y}}}
\newcommand{\setZ}{{\mathfrak{Z}}}

% other commonly defined commands:
\newcommand{\wal}{{\rm wal}}
\newcommand{\floor}[1]{\left\lfloor #1 \right\rfloor} % floor
\newcommand{\ceil}[1]{\left\lceil #1 \right\rceil}    % ceil
\DeclareMathOperator{\cov}{Cov}
\DeclareMathOperator{\var}{Var}
\providecommand{\argmin}{\operatorname*{argmin}}
\providecommand{\argmax}{\operatorname*{argmax}}

\def\bxi{\boldsymbol\xi}
\def\bomega{\boldsymbol\omega}
\def\aff{\operatorname{aff}}
\def\cD{\mathcal{D}}
\def\cS{\mathcal{S}}
\def\cX{\mathcal{X}}


\title[Stochastic Lagrangian]{Stochastic programming\\Lagrangian methods}

\author[Fabian Bastin]{Fabian Bastin \\ \url{bastin@iro.umontreal.ca} \\ Université de Montréal -- CIRRELT -- IVADO -- Fin-ML}

\date{}

\setbeamertemplate{footline}[frame number]

\begin{document}

\frame{\titlepage}

\begin{frame}
\frametitle{Motivation}

{\red Reference}: Birge and Louveaux, Sections~3.4, 5.8 and 6.4.

\mbox{}

We consider the general {\blue two-stage nonlinear program}
\begin{align*}
\inf z\ &= f^1(x) + \mathcal{Q}(x) \\
\mbox{s.t. } & g_i^1(x) \leq 0,\ i = 1,\ldots,m_1,
\end{align*}
where $\mathcal{Q}(x) = E_{\bxi}[Q(x,\bxi)]$, and
\begin{align*}
Q(x,\xi) = \inf\ & f^2(y(\xi),\xi) \\
\mbox{t.q. } & b_i^2(x,\xi) + g_i^2(y(\xi), \xi) \leq 0,\ i = 1,\ldots,m_2.
\end{align*}

We assume that all functions $f^2(\cdot,\xi)$, $b_i^2(\cdot,\xi)$ et $g_i^2(\cdot,\xi)$ are continuous in the first argument for any given $\xi$, and measurable in $\xi$ for any fixed first argument. $\xi$ has finite second-order moments.

\end{frame}

%\begin{frame}
%\frametitle{Application to stochastic programming}

%Consider the stochastic problem under the form
%\begin{align*}
%\inf\ & z = f^1(x) + E_{\bxi}[f^2(y(\xi),\xi)]\\
%\text{s.t. }& g_i^1(x) \leq 0,\ i=1,\ldots{},m_1,\\
%& b_i^2(x,\xi)+g_i^2(y(\xi),\xi) \leq 0, i=1,\ldots{},m_2\ a.s,
%\end{align*}
%where \textit{a.s.} means almost surely.

%\end{frame}

\begin{frame}
	\frametitle{Properties - assumptions}
	
	We can extend the definitions of $K_1$ and $K_2$.
	\begin{align*}
		K_1 &= \lbrace x \,|\, g_i^1(x) \leq 0,\ i = 1,\ldots,m_1 \rbrace,\\
		K_2(\xi) &= \lbrace x \,|\, \exists y(\xi) \mbox{ t.q. } b_i^2(x,\xi) +
		g_i^2(y(\xi),\xi) \leq 0,\ i = 1,\ldots,m_2 \rbrace,\\
		K_2 &= \lbrace x \,|\, \mathcal{Q}(x) < \infty \rbrace.
	\end{align*}
	
	{\blue Assumptions}.
	\begin{enumerate}
		\item \mbox{\red Convexity}.
		The function $f^1$ is convex on $\RR^{n_1}$, $g_i^1$ is convex on $\RR^{n_1}$ ($i =
		1,\ldots,m_1$), $f^2(\cdot,\xi)$ is convex on $\RR^{n_2}$ for all $\xi \in \Xi$, $g_i^2(\cdot,\xi)$ is convex on $\RR^{n_2}$ ($i = 1,\ldots,m_2$), for all $\xi \in \Xi$, and $b_i^2(\cdot, \xi)$ is convex on $\RR^{n_1}$ ($i = 1,\ldots,m_2$), for all $\xi \in \Xi$.
		\item \mbox{\red Slater condition}
		If $\mathcal{Q}(x) < \infty$, for almost every (a.e.) $\xi \in \Xi$, $\exists$ some $y(\xi)$ s.t. $b_i^2(x, \xi) + g_i^2(y(\xi), \xi) < 0$ ($i = 1,\ldots,m_2)$.
	\end{enumerate}
	
\end{frame}

\begin{frame}
\frametitle{Second-stage properties}

The Slater condition ensures that the \textit{strong duality} holds at the second stage and the KKT conditions are necessary and sufficient.

\mbox{}

\begin{theorem}
Under assumptions 1 and 2, the recourse function $Q(x,\xi)$ is a convex function in $x$ for all $\xi \in \Xi$.
\end{theorem}

\begin{theorem}
If the recourse feasible set (i.e. in $y$, for given $\xi$) is bounded for any $x \in \RR^{n_1}$, then the function $Q(x,\xi)$ is lower semi-continuous in $x$ for all $\xi \in \Xi$.
\end{theorem}

\end{frame}

\begin{frame}
\frametitle{Lower semi-continuity}

We say that a function $f$ is {\blue lower semi-continuous in $x_0$} if for every $\varepsilon > 0$, $\exists$ a neighborhood $U$ of $x_0$ such that $f(x) > f(x_0) - \varepsilon$ for all $x \in U$.
Equivalently
\[
\liminf_{x \to x_{0}} f(x) \geq f(x_{0}),
\]
where
\[
\liminf_{x \to a} f(x) = \lim_{\varepsilon \to 0} \inf \{ f(x) : x
\in \mbox{dom}(f) \cap B(a;\varepsilon) \backslash \{a\} \}.
\]

\mbox{}

The function $f$ is {\blue lower semi-continuous} if it is lower semi-continuous in any point in its domain.

\end{frame}

\begin{frame}
\frametitle{Lower semi-continuity}


A function is lower semi-continuous iff $\lbrace x\in X : f(x) > \alpha \rbrace$ is an open set for every  $\alpha \in \RR$, or, in a similar way, $\lbrace x\in X : f(x) \leq \alpha \rbrace$ is a closed set for every $\alpha \in \RR$.

\mbox{}

Example of lower semi-continuous function
$$
f(x) =
\begin{cases}
x & \text{if } x \leq 1 \\
x + 1 & \text{if } x > 1
\end{cases}
$$

\end{frame}

\begin{frame}
\frametitle{Convexity (cont'd)}

\begin{corollary}
The expected recourse function $\mathcal{Q}(x)$ is a convex function in $x$.
\end{corollary}

\begin{corollary}
The feasible set $K_2 = \lbrace x \,|\, \mathcal{Q}(x) < \infty \rbrace$ is closed and convex.
\end{corollary}

\begin{corollary}
If the recourse feasible set if bounded for any $x \in \RR^{n_1}$, $\mathcal{Q}$ is a lower semi-continuous function.
\end{corollary}

\end{frame}

\begin{frame}
\frametitle{Solution: existence}

\begin{theorem}
	If the recourse feasible set is bounded for any $x \in \RR^{n_1}$, $K_1$ is bounded, $f^1$ is continuous, $g_i^1$ and $g_i^2$ are continuous for every $i$, and $K_1 \cap K_2 \ne \emptyset$,
	then the nonlinear stochastic two-stage program has an optimal solution and the infimum is reached.
\end{theorem}

\end{frame}

\begin{frame}
\frametitle{Solution: optimality}

\begin{theorem}
Assume that the Slater condition is satisfied:
$$
 \exists\, x \text{ s.t. } x \in ri(\mbox{dom}(f^1(x)), x \in ri(\mbox{dom}(\mathcal{Q}(x))) \text{ and } g_i^1(x) < 0, i = 1,\ldots,m_1.
$$
$x^*$ is optimal iff there exists $\lambda^*$ such that $(x^*, \lambda^*)$ satisfies the KKT conditions for the two-stage stochastic program, i.e.
\begin{align*}
\bullet\ & x^* \in K_1, \\
\bullet\ & \lambda^* \geq 0, \\ 
\bullet\ & \lambda_i^*g_i^1(x^*) = 0, i = 1,\ldots,m, \\
\bullet\ & 0 \in \partial f^1(x^*) + \partial \mathcal{Q}(x^*) +
 \sum_{i=1}^{m_1}\lambda_i^* \partial g_i^1(x^*).
\end{align*}
\end{theorem}

\end{frame}

\begin{frame}
	\frametitle{Relative interior}
	
	The {\red relative interior} of a set $S$, denoted by $ri(S)$, is defined as its interior within the affine envelop of $S$.
	In other words
	\[
	ri(S) = \lbrace x \in S \,|\, \exists\ \varepsilon > 0, (B_{\varepsilon}(x)
	\cap \aff(S)) \subseteq S \rbrace,
	\]
	where $\aff(S)$ is the {\red affine envelop} of $S$, and $B_{\varepsilon}(x)$
	is a ball of radius ${\varepsilon}$ centered at $x$.
	Any metric can be used for the ball construction: all define the same relative interior.
	
	\mbox{}
	
	The {\red affine envelop} $\aff(S)$ of $S$ is the set of all affine combinations of elements of $S$, i.e.
	\[
	\operatorname{aff} (S)=\left\lbrace \sum_{i=1}^k \alpha_i x_i \,\bigg|\,
	x_i\in S, \, \alpha_i\in \RR, \, \sum_{i=1}^k \alpha_i=1,
	k=1, 2, \dots\right\rbrace. 
	\]
\end{frame}

\begin{frame}
\frametitle{Basic principle of Lagrangian approach}

\begin{itemize}
\item
Nonlinear problem.
\item
We aim to build a search direction and compute a step along this direction to reduce the objective.
\item
Problem: classical nonlinear methods assume that the (sub-)gradients of $\mathcal{Q}$ are available and cheap to obtain.
Not the case here.
\item
Link the first and second stages in order to avoid optimization subproblems when building search directions.
\end{itemize}

\end{frame}

\begin{frame}
\frametitle{Basic principle of Lagrangian approach (cont'd)}

\begin{itemize}
	\item 
Let $\pi$ be a vector of (dual) multipliers associated to the second-stage constraints.
\item
We can form the dual problem with respct to the second-stage constraints (linking the first and second stages) as follows:
\[
\max_{\pi(\xi) \geq 0} w = \theta(\pi),
\]
where
\begin{align*}
\theta(\pi) = \inf_{x, {\bf y}}\ & f^1(x)+E_{\bxi} [f^2(y(\xi),\xi)] + \\
& E_{\xi}\left[\sum_{i=1}^{m_2} \pi_i(\xi) \left( b_i^2(x,\xi) +
  g_i^2(y(\xi), \xi) \right) \right] \\
\text{s.t. } & g_i(x) \leq 0, i=1,\ldots{},m_1.
\end{align*}
\item
We can see that we only consider the dual based on 
We will establish some duality properties in case of finite distribution.
\end{itemize}

\end{frame}

\begin{frame}
\frametitle{Duality}

\begin{theorem}
Assume that all the functions of the stochastic program are convex,
and there exists a finite optimal value, as well as a point strictly satisfying all the constraints.
Assume moreover that the support of $\bxi$ is finite.

\mbox{}

Then $z \geq w$ for all $x, y_1,\ldots{},y_S$ feasible in the primal formulation and $\pi_1,\ldots{},\pi_S$ feasible in the dual formulation (weak duality).

\mbox{}

Moreover, their optimal values coincide: $z^* = w^*$ (strong duality).
\end{theorem}

\mbox{}

We thus assume from now that {\red $\Xi$ is finite}, and we will describe some procedures exploiting the Lagrangian function.

\end{frame}

\begin{frame}
\frametitle{Basic Lagrangian dual ascent method}

Assumption: the dual problem always has a unique solution.

\begin{algo}{Lagrangian dual ascent method}
\begin{description}
\item[Step 0.] Let $\pi^{0,s} \geq 0$, $s = 1,\ldots,S$, $\pi^0 = (\pi^{0,1},\ldots,\pi^{0,S})$, $\nu = 0$.
\item[Step 1.]
Given $\pi = \pi^{\nu}$ in the dual problem, consider the solution
$(x^{\nu}, y_1^{\nu},\ldots{},y_S^{\nu})$.
For $s = 1, \ldots{}, S$, define
$$
\hat{\pi}_i^s = b_i^2(x^{\nu},s)+g_i^2(y_s^{\nu},s).
$$
If $\hat{\pi}^s = 0$ for all $s$, stop.
\item[Step 2.]
Choose a step $\alpha^{\nu} \geq 0$ and set
$
\pi^{\nu+1,s} = \max\{\pi^{\nu,s} + \alpha^{\nu}\hat{\pi}, 0\}.
$
Set $\nu := \nu+1$ and check convergence. If we have not yet converged, return to Step 1.
\end{description}
\end{algo}

\end{frame}

\begin{frame}
\frametitle{Properties - convergence}

\begin{itemize}
\item
Various strategies can be used to compute the step $\alpha^\nu$. Assume we compute the step to maximize $\theta(\pi^{\nu}+\alpha\hat{\pi})$ on $\alpha \geq 0$.
Under the assumption of unicity of dual solution, we can show that this algorithm always produces an {\blue ascent direction} in $\theta$.
\item
Either the algorithm converges to an optimal solution, or, assuming a bounded set of optimal solutions, it produces an infinite sequence where all limit points are optimal.
\item
Good performance if the number of dual iterations is small compared to the number of function evaluations that would be required by solving the original problem directly.
\item
Solving the dual program may be time-consuming, but should nevertheless be easier than solving the original problem directly, as the constraints linking the two stages (i.e. involving $x$ and $y(\xi)$) now appear in the objective.
\end{itemize}

\end{frame}

\begin{frame}

\frametitle{Another approach: scenarios}

As $\Xi$ is finite, we can consider that $S$ scenarios exist, and we can link a non-anticipative decision $\hat{x}$ and the decisions of the scenario $s$ with the terms $(\hat{x}-x_k)$ in the objective function.

\mbox{}

Consider the general problem
\begin{align*}
\inf_{x \in \mathcal{N}}\ & E \left[ \sum_{t=0}^T f_{t+1}(\xi, x^t(\xi),
x_{t+1}(\xi)) \right], \\
\mbox{s.t. } & x_t(\xi) \in X_t(\xi) \mbox{ a.s.},
\end{align*}
where $x_t$ is the decision at stage $t$, and $x^t$ is the decisions history at stage $t$, i.e. $x^t = \lbrace x_1,\ldots, x_t \rbrace$.
$\mathcal{N}$ designs the close linear subspace of nonanticipative processes.
Denote $\mathcal{A}$ the $\sigma$-field of all the events (i.e. the collection of events subsets, with the probability measure associated to $\bxi$).

\end{frame}

\begin{frame}
\frametitle{Nonanticipativity}

\begin{minipage}{0.45\textwidth}
\begin{center}
\hspace*{1cm} \scalebox{0.65}{
\pstree[treemode=R]{\TC}{
  \TC~[tnpos=r]{\quad\bf scenario 1}
  \TC~[tnpos=r]{\quad\bf scenario 2}
  \TC~[tnpos=r]{\quad\bf scenario 3}
  \TC~[tnpos=r]{\quad\bf scenario 4}
 }
}
\end{center}
\end{minipage}
\hfill
\begin{minipage}{0.45\textwidth}
%\setlength{\unitlength}{0.69mm}
\begin{center}
\scalebox{0.7}{
\begin{pspicture}(3,2)(3,6.5)
\pscircle[linewidth=1pt](0,6){0.2}
\psline(0.21,6)(1.79,6)
\pscircle[linewidth=1pt](2,6){0.2}
\psline(-0.04,5.76)(-0.04,5.24)
\psline(0.04,5.76)(0.04,5.24)
\pscircle[linewidth=1pt](0,5){0.2}
\psline(0.21,5)(1.79,5)
\pscircle[linewidth=1pt](2,5){0.2}
\psline(-0.04,4.76)(-0.04,4.24)
\psline(0.04,4.76)(0.04,4.24)
\pscircle[linewidth=1pt](0,4){0.2}
\psline(0.21,4)(1.79,4)
\pscircle[linewidth=1pt](2,4){0.2}
\psline(-0.04,3.76)(-0.04,3.24)
\psline(0.04,3.76)(0.04,3.24)
\pscircle[linewidth=1pt](0,3){0.2}
\psline(0.21,3)(1.79,3)
\pscircle[linewidth=1pt](2,3){0.2}
\end{pspicture}}
\end{center}
\end{minipage}

Mathematically,
\begin{itemize}
\item
the nonanticipativity aims to require that $x^t(\xi)$ has to be $\mathcal{A}^t$-measurable, where $\mathcal{A}^t$ is the $\sigma$-field of events until time $t$;
\item
or, in other words, $x^t(\xi) = E[x^t(\xi) \,|\, \mathcal{A}^t]$ a.s., $t = 0,\ldots$;
\item
using the projection operator $\Pi^t: z \rightarrow \Pi^t z := E[ z | \mathcal{A}^t]$, this is equivalent to
\[
(I-\Pi^t)x^T = 0,\ t = 0,\ldots
\]
\end{itemize}

\end{frame}

\begin{frame}
\frametitle{Nonanticipativity (cont'd)}

Let $S_s^t$ be the set of scenarios identical to the  scenario $s$ at time $t$.
We then have to ensure that
\[
x_{its} = x_{its'},\ \forall i \in N,\ \forall t \in T,\ \forall s \in
S,\ \forall s' \in S_s^t.
\]
In other words, the scenarios share identical decisions as long as they share the same history, i.e. the observed realizations until a given stage are the same.
As soon as the scenarios diverge, the associated decisions can be different.

\mbox{}

{\red Idea}: explicitly include the nonanticipativity constraints, and place them in the objective function, rather than in the second stage constraints.

\end{frame}

\begin{frame}
\frametitle{Dual ascent and scenarios}

Consider the maximization of the dual built on the nonanticipativity constraints:
\[
\max_{\pi(\xi) \geq 0} w = \theta(\pi),
\]
where
\begin{align*}
\theta(\pi) = \inf_{x \in \mathcal{X}} z = & 
E\left[\sum_{t = 0}^T f_{t+1} (\xi, x^t(\xi), x_{t+1}(\xi) \right] \\
& + E\left[\sum_{t = 0}^T \pi^t(\xi)(I-\Pi_t)x^t(\xi) \right],
\end{align*}
where $\pi^t$ corresponds to the components of $\pi$ associated to the $t$ first periods.

\end{frame}

\begin{frame}
\frametitle{Dual ascent and scenarios: two stages}

We still have to describe the projection operator.
For simplicity, consider two steps.
The primal problem becomes (omitting all the constraints except the nonanticipativity ones)
\begin{align*}
\min z\ & = \sum_{s = 1}^S p_s \left( f^1(x_s)+f^2(x_s, y_s) \right) \\
\mbox{s.t. } & x_s - \sum_{k=1}^S p_k x_k = 0,\ s=1,\ldots,S. 
\end{align*}
The dual problem can now be written as
\begin{align*}
\max_{\pi} \theta(\pi) = \min_{x,y} \sum_{s=1}^S p_s\Biggl(& f^1(x_s)+f^2(x_s,
y_s))\\
& + \pi_s\left( x_s - \sum_{k=1}^S p_kx_k \right) \Biggr).
\end{align*}

\end{frame}

\begin{frame}
\frametitle{Dual ascent and scenarios: algorithm}

\begin{algo}{Lagrangian dual ascent method}
\begin{description}
\item[Step 0.] Let $\pi^0 \geq 0$, $\nu = 0$.
\item[Step 1.]
Given $\pi = \pi^{\nu}$ in the dual problem, compute the solution
$(x^{\nu}_1, \ldots, x^{\nu}_S, y_1^{\nu},\ldots{},y_S^{\nu})$.
\item[Step 2.]
If $x_s - \sum_{k = 1}^S p_kx_k = 0$, $s = 1,\ldots,S$, stop: the solution is optimal.

Otherwise, define $\hat{\pi}_s = x_s - \sum_{k=1}^S p_kx_k$, and go to Step 3.
\item[Step 3.]
Let $\alpha^\nu$ maximize $\theta(\pi^{\nu}+\alpha\hat{\pi})$ on
$\pi^{\nu}+\alpha\hat{\pi} \geq 0,\ \alpha \geq 0$.
Set $\pi^{\nu+1} = \pi^{\nu} + \alpha^{\nu}\hat{\pi}$, $\nu = \nu+1$. Return to Step 1.
\end{description}
\end{algo}

\end{frame}

\begin{frame}
\frametitle{Augmented Lagrangian}

Unfortunately, this type of procedure is usually slow as there is a linearization at a single point of $\theta$ only.
It is nevertheless easy to implement and can give good results, especially for small size problems.

\mbox{}

In order to improve the performances, we will turn to another technique inspired by nonlinear programming: {\blue augmented Lagrangian} approaches.

\mbox{}

The basic idea in an augmented Lagrangian procedure is to add a penalty on $\theta(\pi)$ and to build iterations by including this term.

\mbox{}

Moreover, with augmented Lagrangian techniques, we can try to exploit the problem structure in order to decompose the problem.

\end{frame}

%\begin{frame}
%\frametitle{Augmented Lagrangien}

%A way to limit the difficulties associated to penalty methods is to add the penalization term to the Lagrangian rather than the objective, leading to the {\red Augmented Lagrangian} function:
%\[
%\mathcal{L}_A(x,\lambda;\mu) \overset{def}{=} f(x) + \sum_{i \in
%\mathcal{E}} \lambda_i c_i(x) + \frac{1}{2\mu} \sum_{i \in
%\mathcal{E}} c_i^2(x).
%\]

%\mbox{}

%We will then minimize $\mathcal{L}_A(x,\lambda;\mu)$ with respect to $x$, and use some rules to update the Lagrange multipliers, at each iteration.

%\end{frame}

\begin{frame}
\frametitle{Augmented Lagrangian and Stochastic Programming}

We again develop the ideas in the two-stages context, while the approach can be easily generalized in the multi-stage context.

\mbox{}

Recall that we aim to solve
\[
\max_{\pi(\xi) \geq 0} w = \theta(\pi),
\]
where
\begin{align*}
\theta(\pi) = \inf_{x \in \mathcal{X}} z = & 
E\left[\sum_{t = 0}^T f_{t+1} (\xi, x^t(\xi), x_{t+1}(\xi) \right] \\
& + E\left[\sum_{t = 0}^T \pi^t(\xi)(I-\Pi_t)x^t(\xi) \right],
\end{align*}

Penalizing the nonanticipativity constraints, we obtain the following program, where $\hat{x}$ is nonanticipative.

\end{frame}

\begin{frame}
\frametitle{Augmented Lagrangian and SP:: two stages}

\vspace*{-0.5cm}

\begin{align*}
\theta(\rho) = & \inf z = f^1(\hat{x}) + \sum_{s=1}^{S} \left(
  p_sf^2(y_s,\xi_s) + \rho_s^T(x_s-\hat{x}) + \frac{r}{2} \|
x_s-\hat{x} \|^2\right) \\
& \text{s.t. } g_i^1(\hat{x}) \leq 0,\ i=1,\ldots{},m_1,\\
& \phantom{\text{s.t. }} t_i^2(x_s,\xi_s)+g_i^2(y_s,\xi_S) \leq 0,\ 
i=1,\ldots{},m_2,\ s=1,\ldots{},S.
\end{align*}
$r$ if the augmented Lagrangian penalty parameter, and $\rho_s$ is the vector of dual variables associated to the nonanticipativity constraints of the scenario  $s$.

\mbox{}

The method that we will develop aims to contract the pair $(\hat{x}^{\nu+1}, \rho^{\nu+1})$ around a saddle point.

\end{frame}

\begin{frame}
\frametitle{Progressive hedging}

\begin{itemize}
\item
Main reference:
R.~T. Rockafellar and R.~J.-B. Wets, {\sl Scenarios and policy aggregation
in optimization under uncertainty}, Mathematics of Operations
  Research 16(1):119--147 (1991).\\
  The paper focussed on the multistage version.
\item
The method performs a complete separation of the problems, scenario by scenario, at each iteration, reducing the cost per iteration.
\item
The number of iterations can however increase\ldots
\item
But for structures problems, it is possible to exploit parallelism, and therefore solve large-scale problems.
\item
{\red Observation}:
the Lagrangian function is not separable with respect to the scenarios due to the term $(\hat{x}-x_s)$.
\end{itemize}

\end{frame}

\begin{frame}
\frametitle{Progressing hedging: basic principle}

Alternatively,
\begin{enumerate}
\item
fix $\hat{x}$ and obtain the solutions $x_s$, $s = 1,\ldots,S$,
\item
fix $x_s$, $s = 1,\ldots,S$, and compute $\hat{x}$.
\end{enumerate}
In other words, we work scenario by scenario, and we enforce progressively the nonanticipativity constraints.

\mbox{}

At iteration $\nu$, we solve the subproblems
\begin{align*}
\inf z = & \sum_{s=1}^S p_s\left( f^1(x_s) + f^2\left(y_s,\xi_s\right) +
\left(\rho_s^{\nu}\right)^T\left(x_s-\hat{x}^{\nu}\right)+\frac{r}{2}\|
x_s-\hat{x}^{\nu} \|^2\right) \\
& \text{s.t. } g_i^1(x_s) \leq 0,\ i=1,\ldots{},m_1,\ s=1,\ldots{},S,\\
& \phantom{\text{t.q. }}
t_i^2\left(x_s,k\right)+g_i^2\left(y_s,\xi_s\right) \leq 0,\
i=1,\ldots{},m_2,\ s=1,\ldots{},S.
\end{align*}
$\hat{x}^{\nu}$ is not necessarily a feasible solution!

\end{frame}

\begin{frame}
\frametitle{Decomposition}

However, it is easy to decompose the problem!

Algorithm:
\begin{description}
\item[Step 0.]
Assume that we have a nonanticipative solution $x^0$, an initial vector of multipliers $\rho^0$, and $r > 0$.
  $\nu \leftarrow 0$. Go to Step 1.
\item[Step 1.]
Let $\left(x_s^{\nu+1}, y_s^{\nu+1}\right)$ be a solution of the previous Lagrangian program, for $s=1,\ldots{},S$. Set
\[
\hat{x}^{\nu+1} = \left( \hat{x}_1^{\nu+1}, \ldots{}, \hat{x}_S^{\nu+1}\right)^T,
\]
\[
\mbox{where } \hat{x}_s^{\nu+1} = \sum_{s=1}^S p_s x_s^{\nu+1},\ s=1,\ldots,S.
\]
\end{description}

\end{frame}

\begin{frame}
\frametitle{Decomposition (cont'd)}

\begin{description}
\item[Step 2.]
Let $\rho^{\nu+1}_s = \rho^{\nu}_s + r\left(x_s^{\nu+1}-\hat{x}_s
^{\nu+1}\right)$, $s = 1,\ldots{}, S$.\\
If $\hat{x}^{\nu+1} = \hat{x}^{\nu}$ and $\rho^{\nu+1} = \rho^{\nu}$,
stop: $(\hat{x}^{\nu}, \rho^{\nu})$ is optimal.\\
Otherwise, set $\nu = \nu+1$ and return to Step 1.
\end{description}

\mbox{}

\begin{itemize}
\item
Step 2 simply consists to take the expected value of $x^{\nu+1}$
as $\hat{x}^{\nu+1}$.
\item
The basis of this approach is therefore the contraction of the pair $(\hat{x}^{\nu}, \rho^{\nu})$ around a saddle point rather than a dual ascent strategy.
\item
In practice, the optimality test is replaced by a convergence test for instance if the following quantity is small enough (De Silva et Abramson):
\[
\sqrt{ \| \hat{x}^{\nu} - \hat{x}^{\nu+1} \|^2 + \sum_{s = 1}^S p_s \|
  x_s^{\nu} - \hat{x}^{\nu}\|^2}.
\]
\end{itemize}

\end{frame}

\begin{frame}
\frametitle{Generalization to multistage}

$$
\begin{aligned}
\min_{x}\; & \sum_{s \in \cS} p_s f\left(x^{(s)}, \xi^{(s)}\right)\\
\text{s.t. } & x^{(s)} \in \cX^{(s)},\\
& x_t^{(s)} \mbox{ is nonanticipative},\; t = 1, \dots, T.
\end{aligned}
$$

Nonanticipativity:
$$
x_t^{(s)} = E\left[ x_t^{(s')} \,\big|\, s' \in \cS_t^{(s)} \right],
$$
where $\cS_t^{(s)} = \left\{ s' \,\big|\, \bar{\xi}_t^{(s)} = \bar{\xi}_t^{(s') }\right\}$.
It can be reformulated as
\[
x_t^{(s)} = \hat{x}_t^{(s)},
\]
with
$$
\hat{x}_t^{(s)} = \frac{\sum_{s' \in \cS_{t}^{(s)}} p_{s'} x_t^{(s')}}{\sum_{s' \in \cS_t^{(s)}} p_{s'}}.
$$

\end{frame}

\begin{frame}
\frametitle{Generalization}

Augmented Lagrangian:
\begin{align*}
L(x,\lambda,\rho) =  E \Biggl[ & f\left( x^{(s)}, \xi^{(s)} \right) + \sum_{t = 1}^T \biggl( {\lambda_t^{(s)}}' \left(x_t^{(s)} - \hat{x}_t^{(s)}\right) \\
& + \frac{\rho}{2} \left\| x_t^{(s)} - \hat{x}_t^{(s)} \right\|^2 \biggr) \Biggr],
\end{align*}
where $\lambda$ is the Lagrange multipliers vector associated to the NA constraints, and $\rho > 0$ is a penalty parameter.

\end{frame}

\begin{frame}[fragile]
\frametitle{Implementation}

The \verb|StochasticPrograms.jl| package proposes an implementation of the progressive hedging algorithm, but limited to two stages.

\mbox{}

Some variations or non-maintained versions exist.

\end{frame}

\begin{frame}
\frametitle{Generalization}

Given $\lambda$, the augmented Lagrangian program is
\begin{equation}
\begin{aligned}
\min_{x} & \; L(x,\lambda,\rho)\\
\text{s.t.} & \; x^{(s)} \in \mathcal{X}^{(s)}, \, s \in \mathcal{S}.\\
\end{aligned}
\label{Gen-PHA-Form}
\end{equation}
In order to achieve full separability, fix $\hat{x}_t^{(s)}$ and repeatedly solve the program by updating the Lagrange multipliers vector and the value of $\hat{x}_t^{(s)}$ between consecutive resolutions.

\end{frame}

\begin{frame}
\frametitle{Progressive hedging algorithm (PHA)}

\begin{description}
	\item[Step 0.]
	Set $\hat{x}^{(s) ,0} = \left(\hat{x}_1^{(s),0}, \dots, \hat{x}_T^{(s),0}\right)$ and $k = 0$. Choose $\lambda^{(s),0} = \bf{0}$, $\rho^0 > 0$.
	\item[Step 1.]
	Compute $x^{(s),k+1} =  (x_1^{s,k+1}, \dots, x_T^{s,k+1})$, $s = 1,\ldots,S$, by solving each scenario subproblem
	$$
	%\begin{equation}
	%\label{PHA-Scen-subprob}
	\begin{aligned}
	\min_{x^{(s)}} \ &  f\left( x^{(s)}, \xi^{(s)} \right) + \sum_{t = 1}^T \left( {\lambda_t^{(s)}}' \biggl(x_t^{(s)} - \hat{x}_t^{(s),k}\right) \\
	& + \frac{\rho^k}{2} \left\| x_t^{(s)} - \hat{x}_t^{(s),k} \right\|^2 \biggr)\\
	\text{s.t.} & \; x^{(s)}  \in \mathcal{X}^{(s)}.
	\end{aligned}
	$$
	%\end{equation}
	\item[Step 2.]
	For $s = 1,\ldots,S$, $t = 1,\ldots,T$, set
	\[
	\hat{x}_t^{(s),k+1} = \frac{\sum_{s' \in \cS_{t}^{(s)}} p_{s'} x_t^{(s'),k+1}}{\sum_{s' \in \cS_t^{(s)}} p_{s'}}.
	\]
\end{description}

\end{frame}

\begin{frame}
\frametitle{Progressive hedging algorithm (PHA)}

\begin{description}
	\item[Step 3.]
	Set $\rho^{k+1}$ and
	\[
	\lambda_t^{(s), k+1} =  \lambda_t^{(s),k} + \rho^k \left( x_t^{(s),k+1} -  \hat{x}_t^{(s),k+1}  \right),
	\]
	for $t = 1, \dots, T, \, s \in \mathcal{S}$.
	\item[Step 4.]
	Stop if convergence is achieved.
	Otherwise, set $k \leftarrow k + 1$ and return to Step 1.
\end{description}

\end{frame}

\begin{frame}
\frametitle{Convergence}

\begin{theorem}
Assume that the initial nonlinear stochastic program has only convex functions and a finite optimal value, as well as a strictly feasible point.
Assume moreover that the support of $\bxi$ is finite.
The progressive hedging algorithm converges to an optimal solution $x^*, \rho^*$.
\end{theorem}

\begin{itemize}
\item
De Silva and Abramson have tested the progressive hedging algorithm for linear problems in portfolio management, proposed by Mulvey and Vladimirou.
\item
The subproblems where solved by means of interior points methods.
\item
Good speed-up convergence, but the algorithm remains sensitive to the choice of the penality parameter.
\end{itemize}

\end{frame}

\end{document}
