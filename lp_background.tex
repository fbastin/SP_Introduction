\documentclass{beamer}
%\documentclass[xcolor=pst,dvips,epic,eepic]{beamer}

\usepackage[utf8]{inputenc}
\usetheme{Singapore}
\usepackage{xcolor}
\setbeamertemplate{footline}[frame number]

%\usepackage{pgf,pgfarrows,pgfnodes,pgfautomata,pgfheaps,pgfshade}
%\usepackage[pst]{xcolor}
%\usepackage{xxcolor}
\usepackage{mathlist}

%\usepackage[frenchb]{babel}
\usepackage[utf8]{inputenc}
\usepackage{amsmath,amssymb,amsthm}

\newtheorem{theo}{Theorem}

\usepackage{pstricks}

%\usepackage{eepic}

\usepackage{times}

%\usepackage{pst-tree}

\usepackage{ulem}

\usepackage{listings}
%\topmargin=-0.6in
\lstloadlanguages{C++}
\lstset{language=C++}
%%}

\setbeamercovered{dynamic}

\def\bxi{\boldsymbol\xi}
\def\bomega{\boldsymbol\omega}
\def\bxi{\boldsymbol\xi}

\def\rit{\mathcal{R}}

\def\cF{\mathcal{F}}
\def\cK{\mathcal{K}}
\def\cB{\mathcal{B}}
\def\cQ{\mathcal{Q}}
\def\RR{\mathcal{R}}

\def\bb{\boldsymbol{b}}
\def\bc{\boldsymbol{c}}
\def\bx{\boldsymbol{x}}
\def\bA{\boldsymbol{A}}
\def\bB{\boldsymbol{B}}
\def\bD{\boldsymbol{D}}

\def\RR{\mathbb{R}}

\author[Fabian Bastin]{Fabian Bastin \\ \url{fabian.bastin@umontreal.ca} \\ Université de Montréal -- CIRRELT -- IVADO -- Fin-ML}
\date{}
\title[Linear programming]{Linear programming\\Background material}

\begin{document}

\frame{\titlepage}

\begin{frame}
\frametitle{Linear program}

Linear program, standard form:
\begin{align*}
\min_x c^Tx \\
\mbox{s.t. } Ax &= b,\\
x &\geq 0,
\end{align*}
with $x, c \in \RR^n$, $b \in \RR^m$, $A \in \RR^{m \times n}$.

\mbox{}

Assumptions: rank($A$) = $m$, $m \leq n$.
\end{frame}

\begin{frame}
\frametitle{Standard form}

Any linear program can be transformed to the standard form.
Consider for instance the program
\[
\min_x c^Tx \mbox{ s.t. } Ax \geq b.
\]
Note that there is no bound constraints on $x$.

\mbox{}

We first add a vector $z$ of surplus variables:
\begin{align*}
\min_x\ & c^Tx \\
\mbox{s.t. } & Ax - z = b,\\
& z \geq 0.
\end{align*}
We next decompose $x$ as the difference of two non-negative variables: $x = x^+-x^-$, where $x^+ = \max \lbrace x,
0 \rbrace \geq 0$, and $x^- = \max \lbrace -x, 0 \rbrace \geq 0$.
\end{frame}

\begin{frame}
\frametitle{Standard form (cont'd)}

The problem becomes
\begin{align*}
\min_x\ &
\begin{pmatrix}
c & -c & 0
\end{pmatrix}
\begin{pmatrix}
x^+ \\ x^- \\ z
\end{pmatrix}, \\
\mbox{ s.t. } &
\begin{pmatrix}
A & -A & -I
\end{pmatrix}
\begin{pmatrix}
x^+ \\ x^- \\ z
\end{pmatrix} = b,
\begin{pmatrix}
x^+ \\ x^- \\ z
\end{pmatrix} \geq 0.
\end{align*}

The inequalities constraints of the form $x \leq u$ or $Ax \leq b$ can be handled by adding slack variables:
\begin{align*}
x \leq u \Leftrightarrow x+w = u,\ w \geq 0, \\
Ax \leq b \Leftrightarrow Ax + y = b,\ y \geq 0. \\
\end{align*}

\end{frame}

\begin{frame}
	\frametitle{Basic solutions}

Without loss of generality, suppose that the first $m$ columns are independent, and form
	\[
	\bA =
	\begin{pmatrix}
		\bB & \bD
	\end{pmatrix}
	\]
	
	\mbox{}
	
	{\it Basic solution}: $\bx = (\bx_b \ 0)$, with $\bB\bx_b = \bb$.\\
	{\it Degenerated basic solution}: if some components of $\bx_b$ are equal to zero. \\
	{\it Feasible basic solution}: basic solution such that $\bA\bx = \bb$ and $\bx \geq 0$.

\end{frame}

\begin{frame}
	\frametitle{Fundamental theorem of linear programming}

Consider an LP under standard form, with $\bA \in \RR^{m \times n}$, and rank($\bA$) = m.
	\begin{itemize}
		\item
		If there exists a feasible solution, then there exists a feasible basic solution.
		\item
		If there exists a feasible optimal solution, then there exists a feasible optimal basic solution.
	\end{itemize}

\end{frame}
\begin{frame}
	\frametitle{Duality}

\begin{itemize}
	\item 
{\red Primal} program
\begin{align*}
	\min_{x \geq 0} \ & c^T x \\
	\mbox{s.t. } & Ax \geq b
\end{align*}
\item
{\red Dual} program
\begin{align*}
	\max_{\lambda \geq 0} \ & \lambda^T b \\
	\mbox{s.t. } & A^T \lambda \leq c
\end{align*}
\item
$A \in \RR^{m \times n}$, $c, x, \in \RR^n$, $\lambda, b \in \RR^m$.
\end{itemize}
\end{frame}

\begin{frame}
	\frametitle{Duality}

\begin{itemize}
	\item 
	$x$: primal variables
	\item
	$\lambda$: dual variables
\item 	
	Dual of the dual:
	\begin{align*}
		\min_x \ & c^T x \\
		\mbox{s.t. } & Ax \geq b \\
		& x \geq 0
	\end{align*}
\end{itemize}
	
\end{frame}

\begin{frame}
	\frametitle{Duality: standard form}
	
	The program
	\begin{align*}
		\min_x \ & c^T x \\
		\mbox{s.t. } & Ax = b \\
		& x \geq 0
	\end{align*}
	is equivalent to
	\begin{align*}
		\min_x \ & c^T x \\
		\mbox{s.t. } & Ax \geq b \\
		& -Ax \geq -b \\
		& x \geq 0
	\end{align*}
	
\end{frame}

\begin{frame}
	\frametitle{Duality: standard form}
	
The dual problem can then be written as
	\begin{align*}
		\max_{u, v} \ & u^T b - v^T b \\
		\mbox{s.t. } & u^T A - v^T A \leq c^T \\
		& u \geq 0 \\
		& v \geq 0
	\end{align*}
	or, with $\lambda = u - v$,
	\begin{align*}
		\max_{\lambda} \ & \lambda^T b \\
		\mbox{s.t. } & \lambda^T A \leq c^T \\
	\end{align*}
	
Asymmetric from $\lambda \in \RR^m$.
	
\end{frame}


\begin{frame}
	\frametitle{Primal-dual conversion}
	
	\begin{center}
		\begin{tabular}{|c|c|}
			\hline
			\hline
			{\bf Minimization} & {\bf Maximization} \\
			\hline
			\hline
			Constraints & Variables \\
			$\geq$ & $\geq 0$ \\
			$\leq$ & $\leq 0$ \\
			$=$ & unconstrained \\
			\hline
			Variables & Constraints \\
			\hline
			$\geq 0$ & $\leq$ \\
			$\leq 0$ & $\geq$ \\
			unconstrained & $=$ \\
			\hline
		\end{tabular}
	\end{center}
	
\end{frame}

\begin{frame}
	\frametitle{Weak duality}
	
	(Asymmetric or symmetric form)
	
	\mbox{}
	
	If $x$ and $\lambda$ are feasible for the primal and the dual, respectively, then
	\[
	c^T x \geq \lambda^T b
	\]
	
	\begin{proof}
		For any primal feasible x and dual feasible $\lambda$, we have
		\[
		\lambda^T b \leq \lambda^TAx \leq c^Tx.
		\]
	\end{proof}
	
\end{frame}

\begin{frame}
	\frametitle{Corollary}
	
	If $x_0$ and $\lambda_0$ are feasible for the primal and the dual, respectively, and if
	\[
	c^T x_0 = \lambda_0^T b,
	\]
	then $x_0$ and $\lambda_0$ are optimal for their respective problem.
	
	\mbox{}
	
	But we have said nothing about the feasibility of one problem with respect to the other one!
	
\end{frame}

\begin{frame}
	\frametitle{Strong duality}
	
	If one of the problems, primal or dual, has a finite optimal solution, the other problem also has a finite optimal solution, and the corresponding values of the objective functions are equal.
	If one of the problems has an unbounded objective, the other problem has no feasible solution.

\mbox{}

If a program is unfeasible, this does not imply that its dual is unbounded. It can also be unfeasible.

\mbox{}

\begin{center}
	\begin{tabular}{|c||c|c|c|}
		\hline
		Primal / Dual & Bounded & Unbounded & Unfeasible \\
		\hline
		\hline
		Bounded & possible & impossible & impossible \\
		\hline
		Unbounded & impossible & impossible & possible \\
		\hline
		Unfeasible & impossible & possible & possible \\
		\hline
	\end{tabular}
\end{center}

\end{frame}

\begin{frame}
\frametitle{Optimality et duality}

The optimality can be deduced from KKT conditions.

\mbox{}

In LP, we do not need a constraint qualification to apply KKT conditions.

\mbox{}

Lagrangian:
\[
\mathcal{L} = c^Tx - \pi^T(Ax-b) - s^Tx.
\]

KKT conditions: Lagrangian vectors $\pi$ and $s$ s.t.
\begin{align*}
A^T\pi + s &= c,\\
Ax &= b,\\
x &\geq 0,\\
s &\geq 0,\\
x_is_i &= 0,\ i=1,2,\ldots,n \quad \rm{(complementarity)}.
\end{align*}

\end{frame}

\begin{frame}
\frametitle{Dual problem}

Let $(x^*, \pi^*, s^*)$ be a vector satisfying the KKT conditions. We have that
\[
c^Tx^* = (A^T\pi^*+s^*)^Tx^* = (Ax^*)^T\pi^* = b^T\pi^*.
\]

It is furthermore easy to show that the (necessary) KKT conditions are sufficient.

\mbox{}

{\red Dual problem}:
\[
\max_\pi b^T\pi, \mbox{ s.t. } A^T\pi \leq c,
\]
or, equivalently,
\[
\max_\pi b^T\pi, \mbox{ s.t. } A^T\pi +s = c,\ s \geq 0.
\]

\end{frame}

\end{document}