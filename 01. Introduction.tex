\documentclass{beamer}
%\documentclass[xcolor=pst,dvips,epic,eepic]{beamer}

\usepackage[utf8]{inputenc}
\usetheme{Singapore}
\usepackage{xcolor}
\setbeamertemplate{footline}[frame number]

%\usepackage{pgf,pgfarrows,pgfnodes,pgfautomata,pgfheaps,pgfshade}
%\usepackage[pst]{xcolor}
%\usepackage{xxcolor}
\usepackage{mathlist}

\usepackage[utf8]{inputenc}
\usepackage{amsmath,amssymb}
\usepackage{colortbl}
%\usepackage[english]{babel}
%\usepackage{pstricks}
%\usepackage{pst-tree}
%\usepackage{pictex}
%\usepackage{tikz}
%\usetikzlibrary{trees}
%\usepackage{tikz-qtree}

\usepackage{times}

%\usepackage{pst-tree}

\usepackage{ulem}

\usepackage{listings}
%\topmargin=-0.6in
\lstloadlanguages{C++}
\lstset{language=C++}
%%}

\setbeamercovered{dynamic}

\include{macros}

\def\eqdef {\buildrel \rm def \over =}    % Egal par definition.
\def\eqas  {\buildrel \rm a.s. \over =}   % Egal a.s.
\def\toas  {\buildrel \rm a.s. \over \to} % ---> a.s.
\def\q     {$\kern1.4em$}    % Indentation spaces for slides, algor, programs.
\def\?{\discretionary{}{}{}}  % Same as \- but does not print the - sign
\def\g   {{\;\leftarrow\;}}

\def\Var{{\rm Var}}
\def\Cov{{\rm Cov}}
\def\MSE{{\rm MSE}}
\def\RE{{\rm RE}}
\def\REff{{\rm REff}}
\def\Eff{{\rm Eff}}
\def\eff{{\rm Eff}}
\def\tr{{\sf T}}
\def\byC {{\bf \yel{C}}}
\def\byh {{\bf \yel{h}}}
\def\byq {{\bf \yel{q}}}
\def\byY {{\bf \yel{Y}}}
\def\byzero {{\bf \yel{0}}}
\def\bySigma {{\bf \yel{\Sigma}}}

\def\blue{\color{blue}}
\def\red{\color{red}}

\title[Stochastic optimization]{Stochastic optimization}

\author[Fabian Bastin]{Fabian Bastin \\ \url{fabian.bastin@umontreal.ca} \\ Université de Montréal -- CIRRELT -- IVADO}
\date{}

\begin{document}

\titlegraphic{\vspace{-0.5cm}
	\includegraphics[scale=0.2]{imgs/udem.png}
%	\hspace{2cm}
%	\includegraphics[scale=0.07]{imgs/Fin-MLLogo.png}
}
\frame{\titlepage}

\begin{frame}
\frametitle{Introduction}

Consider the general deterministic program
\[
\begin{split}
& \min g_0(x) \\
& \text{s.t. }g_i(x) \leq 0, i = 1,\ldots{},m \\
& \phantom{t.q. }x \in X \subset \calR^n.
\end{split}
\]

\mbox{}

All the parameters are assumed to be perfectly known. {\color{red} Realistic?}
\begin{itemize}
\item
measurement errors;
\item
uncertainties on the future;
\item
data unavailable;
\item
\ldots
\end{itemize}

\end{frame}

\begin{frame}
\frametitle{Mathematical programming and stochastic programming}

\begin{itemize}
\item
{\red Mathematical programming} (optimization): typically: decision problem (where the meaning of the term ``decision'' is broad).
\item
{\red Stochastic programming} concerns decision under uncertainty, the uncertainty being represented by means of random parameters.
\end{itemize}
\[
\begin{split}
& ``\min_{x \in X}" g_0(x,\bsxi) \\
& \text{s.t. }g_i(x, \bsxi) \leq 0,\ i=1,\ldots{},m,\\
\end{split}
\]
where $\bsxi$ is a random vector. Meaning of ``$\min$''?

\mbox{}

{\red Assumption}: we can represent the uncertainty by means of the (joint) probability distribution.

\end{frame}

\begin{frame}
\frametitle{The farmer problem}

Source: Birge et Louveaux, Section~1.1.

\mbox{}

{\blue Scenarios approach}
\begin{itemize}
\item
Assumption: finite random vector. A realization = a scenario.
\item
Even if the random vector is continuous, a discrete approximation is often useful.
\end{itemize}

\mbox{}

\begin{itemize}
\item
A European farmer has 500 acres of land.
\item
He cultivates wheat, corn and sugar beets.
\item
Livestock food: at least 200T of wheat and 240T of corn.
\item
Must buy if less production, but can sell the overproduction.
\item
The purchase cost is 40\% greater than the sale cost.
\item
The farmer can sold the sugar beets at \$36T for the first 6000 tons, and \$10T after, due to European quotas.
\end{itemize}

\end{frame}

\begin{frame}
\frametitle{The farmer problem II}

\begin{center}
\begin{tabular}{lccc}
\hline
Culture & Wheat & Corn & Sugar beets \\
\hline
Average return ($T$) & 2.5 & 3 & 20 \\
Plantation cost (\$/acre) & 150 & 230 & 260 \\
Selling price (\$/T) & 170 & 150 & 36 ($\leq$ 6000T), 10 \\
Buying price (\$/T) & 238 & 210 & - \\
Minimum required (T) & 200 & 240 & - \\
\hline
\end{tabular}
\end{center}

\mbox{}

Notations:
\begin{itemize}
\item
$x_1$, $x_2$, $x_3$: acres for wheat, corn, sugar beets;
\item
$y_1$, $y_2$: tons of bought wheat and corn;
\item
$w_1$, $w_2$: tons of sold wheat and corn;
\item
$w_3$, $w_4$: tons of sold sugar beets, at high price and at low price.
\end{itemize}

\mbox{}

How to decide the surface to allocate to each plant?

\end{frame}

\begin{frame}
\frametitle{The farmer problem: deterministic version}

Linear program:
\begin{align*}
\min\ & 150x_1 + 230x_2 + 260x_3 +\\
&  238y_1 - 170w_1 + 210y_2 - 150w_2 -36w_3 - 10w_4 \\
\text{s.t. } & x_1 + x_2 + x_3 \leq 500; \\
& 2.5x_1 + y_1 - w_1 \geq 200; \\
& 3x_2 + y_2 - w_2 \geq 240; \\
& w_3 + w_4 \leq 20 x_3; \\
& w_3 \leq 6000; \\
& x_1, x_2, x_3, y_1, y_2, w_1, w_2, w_3, w_4 \geq 0.
\end{align*}

\end{frame}

\begin{frame}
\frametitle{The farmer problem: deterministic solution}

Total (expected) profit: \$118600. Details:
\begin{center}
\begin{tabular}{lccc}
\hline
Culture & Wheat & Corn & Sugar beets \\
\hline
Surface (acres) & 120 & 80 & 300 \\
Production (T) & 300 & 240 & 6000 \\
Sales (T) & 100 & - & 6000 \\
Purchase (T) & - & - & - \\
\hline
\end{tabular}
\end{center}

\mbox{}

Production can however increase or decrease by 20\% to 25\%, depending on the weather.
%The production is however dependant on the weather, and can .
Simplified setting:
\begin{itemize}
	\item 
good year (for every plant, the production is 20\% higher),
\item
average year,
\item
bad year (for every plant, the production is 20\% lower).
\end{itemize}
The prices do not change.
\end{frame}

\begin{frame}
\frametitle{The farmer problem: scenario solutions}

{\blue New optimal solutions?}

\mbox{}

{\red Good year}. Total profit: \$167667.
\begin{center}
\begin{tabular}{lccc}
\hline
Culture & Wheat & Corn & Sugar beets \\
\hline
Surface (acres) & 183.33 & 66.67 & 250 \\
Production (T) & 550 & 240 & 6000 \\
Sales (T) & 350 & - & 6000 \\
Purchases (T) & - & - & - \\
\hline
\end{tabular}
\end{center}

\mbox{}

\mbox{}

{\red Bad year}. Total profit: \$59950.
\begin{center}
\begin{tabular}{lccc}
\hline
Culture & Wheat & Corn & Sugar beets \\
\hline
Surface (acres) & 100 & 25 & 375 \\
Production (T) & 200 & 60 & 6000 \\
Sales (T) & - & - & 6000 \\
Purchases (T) & - & 180 & - \\
\hline
\end{tabular}
\end{center}

\end{frame}

\begin{frame}
\frametitle{The farmer problem: scenarios}

The decisions considerably change with the weather conditions, but how to know them when deciding what to plant?
\mbox{}

The decisions $(x_1, x_2, x_3)$ have to be made now, but sales and purchases $(w_i, i=1,\ldots,4, y_j, j=1,2)$ depend on yields.

\mbox{}

{\blue Scenarios}: $s \in \{1,2,3\}$.
\begin{description}
	\item[$s = 1$] yields higher than average,
	\item[$s = 2$] equal to average,
	\item[$s = 3$] lower than average.
\end{description}
New variables: $w_{is}$ and $y_{is}$.

\end{frame}

\begin{frame}
\frametitle{The farmer problem: extensive form}

\begin{itemize}
	\item 
We now want to maximize the {\red expected profit}.
\item
Assumption: 3 equiprobable scenarios.
\end{itemize}
\begin{small}
\begin{align*}
\min\ & 150x_1 + 230x_2 + 260x_3 +\\
&  + \sum_{s = 1}^3 \frac{1}{3}(238y_{1s} - 170w_{1s} + 210y_{2s} - 150w_{2s} -
36w_{3s} - 10w_{4s}) \\
\text{t.q. } & x_1 + x_2 + x_3 \leq 500; \\
& 3x_1 + y_{11} - w_{11} \geq 200; 2.5x_1 + y_{12} - w_{12} \geq 200; \\
& \quad{}  2x_1 + y_{13} - w_{13} \geq 200; \\
& 3.6x_2 + y_{21} - w_{21} \geq 240; 3x_2 + y_{22} - w_{22} \geq
240;\\
& \quad{} 2.4x_2 + y_{23} - w_{23} \geq 240; \\
& w_{31} + w_{41} \leq 24 x_3; w_{32} + w_{42} \leq 20 x_3; w_{33} +
w_{43} \leq 16 x_3; \\
& w_{31} \leq 6000; w_{32} \leq 6000; w_{33} \leq 6000; \\
& x, y, w \geq 0.
\end{align*}
\end{small}
$\rightarrow$ {\blue extensive form} (or {\blue deterministic equivalent}).

\end{frame}

\begin{frame}
\frametitle{The farmer problem: stages}

\begin{itemize}
	\item 
Seeding decisions: {\red first-stage decisions};
	\item 
Sale and purchase decisions: {\red second-stage decisions}.
\end{itemize}

Total profit: \$108390.
\begin{center}
\begin{tabular}{llccc}
\hline
& Culture & Wheat & Corn & Sugar beets \\
\hline
First stage & Surface (acres) & 170 & 80 & 250 \\
\hline
$s = 1$ & Productions (T) & 510 & 288 & 6000 \\
& Sales (T) & 310 & 48 & 6000 \\
& Purchases (T) & - & - & - \\
\hline
$s = 2$ & Productions (T) & 425 & 240 & 5000 \\
& Sales (T) & 225 & - & 5000 \\
& Purchases (T) & - & - & - \\
\hline
$s = 3$ & Productions (T) & 340 & 192 & 4000 \\
& Sales (T) & 140 & - & 4000 \\
& Purchases (T) & - & 48 & - \\
\hline
\end{tabular}
\end{center}

\end{frame}

\begin{frame}
\frametitle{Observations}

\begin{itemize}
	\item 
The optimal decision has changed!!!
\item
Decision under {\red perfect information}:
if the farmer could know the scenario in advance, or wait to observe the realization of the random variables (r.v.) ({\red wait-and-see} approach),
the average annual profit would be \$115406.
\item
The difference with the optimal decision under uncertainty is called {\blue expected value of perfect information (EVPI)}: profit loss due to uncertainty. 
\end{itemize}

\end{frame}

\begin{frame}
\frametitle{Observations: value of the stochastic solution}

\begin{itemize}
\item
Replacing $\bsxi$ by $\EE[\bsxi]$ in the problem leads to the expected value (EV) problem, delivering the {\blue expected value solution}.
\item
The {\blue expectation of the expected value (EEV)} problem is obtained by computing the expected profit over the scenarios when the EV solution is always used.
\item
Here, the expectation of scenarios is the average year, but in general, the expectation will not necessarily correspond to a pre-existent scenario.
\item
If the farmer only uses the average information, the average profit would be \$107240.
\item
It leads to a loss of \$1150 with respect to the solution of the stochastic problem. This difference is known as {\red value of the stochastic solution (VSS)}.
\end{itemize}

\end{frame}


\begin{frame}
\frametitle{Example: the newsvendor problem}

Source: Birge and Louveaux, Section~1.1.

\mbox{}

\begin{itemize}
	\item
	A newsvendor has to decide how many newspapers to buy in order to maximize his profit.
	However he does not know in advance how many newspapers he will be able to sell during a day (the demand).
	\item
	Each newspaper costs $c$, and can be sold at a price $q$.
	\item
	The newsvendor can turn back the unsold newspapers at the end of the day, and obtain a price $r$ for each of them
	\item
	Knowing the probability distribution $F(t) = P(\bsxi \leq t$), how many newspapers should the newsvendor buy in order to maximize his revenue?
\end{itemize}

\end{frame}

\begin{frame}
\frametitle{The newsvendor problem (cont'd)}

\begin{itemize}
\item
With the previous definitions, the newsvendor would like to solve the following optimization problem:
\[
\max_{x \geq 0} -cx + \mathcal{Q}(x),
\]
\item
$\mathcal{Q}(x)$ is the expected sale amount if the newsvendor buy $x$ newspapers:
\[
\mathcal{Q}(x) = \EE_{\bsxi} [ Q(x, \bsxi)].
\]
%\item
%Dans le cadre linéaire, nous écrirons aussi souvent
%$Q(x, \omega) = v(h(\omega) - T (\omega)x)$.
\item
Here $Q(x, \xi)$ is the amount of money obtained by the newsvendor if he buys $x$ newspaper and the demand is $\xi$.
\end{itemize}

\end{frame}

\begin{frame}
\frametitle{The newsvendor problem (cont'd)}

\begin{itemize}
\item
As previously, we could construct an equivalent linear problem (presented in Birge and Louveaux).
\item
Can we simplify? It is easy to see that
\[
Q(x, \xi) =
\begin{cases} 
qx & \mbox{if } x \leq \xi, \\
q\xi + r(x - \xi) & \mbox{if } x \geq \xi.
\end{cases}
\]
Therefore,
\begin{align*}
\mathcal{Q}(x) & = E_{\bsxi} [ Q(x, \bsxi) ] =
\int_{-\infty}^{\infty} Q(x, \xi) dF(\xi) 
\\ & = \int_{-\infty}^x(q\xi + r(x - \xi))dF(\xi) + \int^{\infty}_x
qxdF(\xi).
\end{align*}
\end{itemize}

\end{frame}

\begin{frame}
\frametitle{The newsvendor problem (cont'd)}

Therefore, we have
\begin{align*}
\mathcal{Q}(x) & = (q-r) \int_{-\infty}^x \xi dF(\xi) + rx
\int_{-\infty}^x dF(\xi) + qx \int_x^{\infty} dF(\xi) \\
& = (q-r) \int_{-\infty}^x \xi dF(\xi) + rx F(x) + qx (1-F(x)) \\
& = (q-r) \left[ \int_{-\infty}^x \xi dF(\xi) - x F(x) \right] + qx. \\
\end{align*}

\end{frame}

\begin{frame}
\frametitle{Integration by parts}

Assume that $F$ satisfies $\lim_{t \rightarrow -\infty}
tF(t) = 0$.

\mbox{}

We can then integrate by parts to obtain:
\begin{align*}
\int_{-\infty}^x \xi dF(\xi) &= \xi F(\xi) |_{-\infty}^x -
\int_{-\infty}^x F(\xi) d\xi \\
&=  xF(x) - \int_{-\infty}^x F(\xi) d\xi.
\end{align*}

Thus,
\[
\mathcal{Q}(x) = qx - (q - r)\int_{-\infty}^x F(\xi)d\xi.
\]

\end{frame}

\begin{frame}
\frametitle{Solution of the second stage}

Recall the initial problem\ldots
\[
\max_{x \geq 0} -cx + \mathcal{Q}(x),
\]
We have to solve this problem.
We will consider the associated optimality conditions.

\mbox{}

Assuming $x \ne 0$, the solution of the second-stage is obtained by computing the zero of the objective gradient.
As
\[
\frac{d}{dx} \mathcal{Q}(x) = q - (q - r)F(x),
\]
we have
\[
-c + q - (q - r)F(x) = 0
\]

\end{frame}

\begin{frame}[fragile]
\frametitle{The newsvendor problem (cont'd)}

The solution $x^*$ is therefore
\[
x^* = F^{-1} \left( \frac{q-c}{q-r} \right).
\]

\mbox{}

Example: $c = 0.15$, $q = 0.25$, $r = 0.02$, $\bsxi \sim N(650, 80^2)$. Alors
\[
x^* = N^{-1}_{(650,80^2)}(0.1/0.23).
\]
Since $N(650, 80^2) \sim 80\Phi+650$, where $\Phi$ is the distribution function of a $N(0,1)$, it easy to show that
\[
x^* = 80\Phi^{-1}(0.1/0.23) + 650 \approx 636.86.
\]
In Julia, we can compute this value as
\begin{verbatim}
using Distributions
d = Normal(650,80)
quantile(d, 0.1/0.23)
\end{verbatim}

% regarder aussi
%https://github.com/martinbiel/StochasticPrograms.jl

\end{frame}

\begin{frame}
\frametitle{Marginal revenue}

{\sl Other interpretation}, more intuitive: assume that the vendor has bought $t$ journaux.
What is the expected marginal revenue if he buys an additional newspaper?
On an economical point of view, we would like this marginal revenue to be equal to 0.

\mbox{}

The expected marginal revenue (MR) is
\begin{align*}
MR(t) &= -c + qP[\bsxi \geq t] + rP[\bsxi \leq t]\\
&= -c + q(1-F(t)) + rF(t).
\end{align*}
If we set the marginal revenue to 0, we get
$$
MR(t) = 0 \mbox{ iff } F(t) = \frac{q-c}{q-r},
$$
and we recover the previous solution.

%Pour des compléments, consultez Linderoth.

\end{frame}

\end{document}
